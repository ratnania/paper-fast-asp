\section{Auxiliary Space Preconditioners}\label{sec:asp}
The objective of this section is the construction of an appropriate auxiliary space preconditioner for our \textbf{curl}-\textbf{curl} problem. As already mentioned, the main challenge is the derivation of a discrete version of the regular decomposition of Theorem \ref{regular-decomposition-of-H(curl)}, namely the Hitmair-Xu decomposition. The presentation is in three subsections. First, in Subsection \ref{subsec:asp} we focus on the case without mapping where $\Omega$ is simply a parametric domain given by $\Omega=(0,1)^3$ and we show the discrete Hitmair-Xu decomposition in this case. The result of this subsection is extended to the case of a physical domain in Subsection \ref{subsec:asp-with-mapping}. The construction of the preconditioner is developed in Subsection \ref{subsec:construction-of-ASP}.



Through this section $A \lesssim B$ means that there exists some constant $C>0$, which is independent of $h$, such that $A \leq CB$.
 
\subsection{Auxiliary Space Preconditioners}\label{subsec:construction-of-ASP}
Now we have all the ingredients to apply the abstract ASP theory of Section \ref{subsec:Auxiliary-Space-Preconditioning-Method}. Indeed, armed with the above stable discrete decomposition result, using the notations of Section \ref{subsec:Auxiliary-Space-Preconditioning-Method}, we consider $V=\bm{V}_h(\textbf{curl},\Omega)$ equipped with the bilinear form $a$ related to equation \eqref{eq:curl-variational-problem}, namely $a(\bm{w}_h,\widetilde{\bm{w}_h})= \left(\textbf{curl}\,(\bm{w}_h),\textbf{curl}\,(\widetilde{\bm{w}_h})\right)_{\mathbb{L}^2(\Omega)} + \mu (\bm{w}_h,\widetilde{\bm{w}_h})_{\mathbb{L}^2(\Omega)}$ for $\bm{w}_h, \widetilde{\bm{w}_h} \in V$, and auxiliary spaces $W_1= \mathbb{V}_h(\textbf{curl},\Omega)$ and $W_2=V_h(\textbf{grad},\Omega)$ equipped with the following inner products
\begin{equation*}
a_1(\bm{\varphi}_h,\widetilde{\bm{\varphi}_h})= (\bm{\varphi}_h,\widetilde{\bm{\varphi}_h})_{\mathbb{H}^1(\Omega)} + \mu (\bm{\varphi}_h,\widetilde{\bm{\varphi}_h})_{\mathbb{L}^2(\Omega)}, \quad \bm{\varphi}_h,\widetilde{\bm{\varphi}_h} \in W_1,
\end{equation*}
and
\begin{equation*}
a_2(\phi_h,\widetilde{\phi_h})= \mu \, (\textbf{grad}\,(\phi_h),\textbf{grad}\,(\widetilde{\phi_h}))_{\mathbb{L}^2(\Omega)},  \quad \phi_h,\widetilde{\phi_h} \in W_2,
\end{equation*}
respectively. The corresponding transfer operators are $\pi_1 = \Pi_h^{\textbf{curl}}\,:\, W_1 \longrightarrow V$ and $\pi_2=\textbf{grad}\,:\, W_2 \longrightarrow V$ and the smoother is simply the inner product $a$. 

Before we verify the validity of assumptions of Theorem \ref{th:ASP-lemma}, we transition to a matrix notation. So we write $C_h^1$ for the matrix related to the restriction of $\mathbb{H}^1(\Omega)$ inner product to $\mathbb{V}_h(\textbf{grad},\Omega)$, that is the matrix representation of the bilinear form 
\begin{equation*}
(\bm{\varphi}_h,\widetilde{\bm{\varphi}_h}) \in \mathbb{V}_h(\textbf{grad},\Omega) \times \mathbb{V}_h(\textbf{grad},\Omega)  \mapsto (\bm{\varphi}_h,\widetilde{\bm{\varphi}_h})_{\mathbb{H}^1(\Omega)}.
\end{equation*}
Similarly, we write $C_h^2$ for the matrix related to the restriction of the $\mathbb{L}^2(\Omega)$ inner product to $\mathbb{V}_h(\textbf{grad},\Omega)$. Let $L_h$ be the discrete Laplacian matrix, namely the matrix related to the mapping  
\begin{equation*}
(\bm{\phi}_h,\widetilde{\bm{\phi}_h}) \in V_h(\textbf{grad},\Omega) \times V_h(\textbf{grad},\Omega)  \longmapsto \left(\textbf{grad}\,(\bm{\phi}_h),\textbf{grad}\,(\widetilde{\bm{\phi}_h})\right)_{\mathbb{L}^2(\Omega)}.
\end{equation*}
We also write $P_h^{\textbf{curl}}$ and $G_h$  for the matrices related to the transform operators $\pi_1$ and $\pi_2$ respectively while $R_h$ further stands for the matrix related to the smoother. With these notations, a simple computation shows that ASP preconditioner  for problem \eqref{eq:curl-variational-problem} reads 
\begin{equation*}
B_h=R_h^{-1} + P_h^{\textbf{curl}}  \left(C_h^1+\mu C_h^2 \right)^{-1} \left( P_h^{\textbf{curl}}\right)^T + \mu^{-1} G_h L_h^{-1} \left(G_h\right)^T. 
\end{equation*}

A direct consequence of Theorem \ref{thm:stable-Hitpmair-Xu-decomposition} and Theorem \ref{th:ASP-lemma} the following fundamental result which show the mesh-independent of the preconditioner.
\begin{theorem}
For $0 < \mu \leq 1$, the spectral condition number $\kappa \left( B_h A_h\right)$ is bounded, with respect to $h$.
\end{theorem}
