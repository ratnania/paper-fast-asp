\section{Auxiliary Space Preconditioners}\label{sec:asp}
The objective of this section is the construction of an appropriate auxiliary space preconditioner for our \textbf{curl}-\textbf{curl} problem. As already mentioned, the main challenge is the derivation of a discrete version of the regular decomposition of Theorem \ref{regular-decomposition-of-H(curl)}, namely the Hitmair-Xu decomposition. The presentation is in three subsections. First, in Subsection \ref{subsec:asp} we focus on the case without mapping where $\Omega$ is simply a parametric domain given by $\Omega=(0,1)^3$ and we show the discrete Hitmair-Xu decomposition in this case. The result of this subsection is extended to the case of a physical domain in Subsection \ref{subsec:asp-with-mapping}. The construction of the preconditioner is developed in Subsection \ref{subsec:construction-of-ASP}.



Through this section $A \lesssim B$ means that there exists some constant $C>0$, which is independent of $h$, such that $A \leq CB$.
 
\subsection{Discrete Decompositions without Mapping}\label{subsec:asp}
Throughout this subsection, $\Omega=(0,1)^3$.

We need the following preliminary results in order to prove Hitpmair-Xu decomposition stated in Proposition \ref{prop:Hitpmair-Xu decomposition}.

\begin{lemma}\label{prelimibary-lemma-1}
For every $\bm{\varphi} \in \mathbb{H}^1(\Omega)$ such that $\textbf{curl} \, (\bm{\varphi}) \in \bm{V}_h(div,\Omega)$, we have
\begin{itemize}
\item[(i)] $\Pi_h^{\textbf{curl}} \bm{\varphi}$ is well-defined;

\item[(ii)] $\textbf{curl}\,(\bm{\varphi})=\textbf{curl}\left( \Pi_h^{\textbf{curl}} \bm{\varphi}\right)$;

\item[(iii)] $\left\| \bm{\varphi}-\Pi_h^{\textbf{curl}} \bm{\varphi} \right\|_{\mathbb{L}^2(\Omega)} \lesssim h \|\bm{\varphi}\|_{\mathbb{H}^1(\Omega)}$.
\end{itemize}
\end{lemma}

\begin{proof}
First insertion is a consequence of the fact that $\mathbb{H}^1(\Omega) \subset \mathbf{H}(\textbf{curl},\Omega)$. Concerning (ii), using the commutativity of Diagram \eqref{pr-eq:DeRham-diagram}, we obtain
$$
\textbf{curl}\left( \Pi_h^{\textbf{curl}} \bm{\varphi}\right)= \Pi_h^{\text{div}} \left( \textbf{curl}\, (\bm{\varphi}) \right).
$$
We now use $\textbf{curl}\,(\bm{\varphi}) \in \bm{V}_h(div,\Omega)$ to obtain (ii). Estimate (iii) follows from Theorem \cite[Theorem 5.3]{buffa2011isogeometric} with the particular choice of $l=0$ and $s=1$. 
\end{proof}

\begin{lemma}\label{lem-semi-discrete-decomposition}
For each $\bm{u}_h \in \bm{V}_h(\textbf{curl},\Omega)$, there exist $\bm{\varphi} \in \mathbb{H}^1(\Omega)$ and $\phi_h \in V_h(\textbf{grad},\Omega)$ such that
\begin{equation}\label{eq:semi-discrete-decomposition}
\bm{u}_h = \Pi_h^{\textbf{curl}} \bm{\varphi} + \textbf{grad}\,(\phi_h),
\end{equation}
with estimates
\begin{equation}\label{eq:discrete-H-curl-estimate}
\|\bm{\varphi}\|_{\mathbb{H}^1(\Omega)} \lesssim \|\textbf{curl} \,(\bm{u}_h)\|_{\mathbb{L}^2(\Omega)},
\end{equation}
\begin{equation}\label{eq:stability-of-semi-discrete-decomposition}
\|\Pi_h^{\textbf{curl}} \bm{\varphi}\|_{\mathbb{L}^2(\Omega)} + \|\textbf{grad}\,(\phi_h)\|_{\mathbb{L}^2(\Omega)}\lesssim \|\bm{u}_h\|_{\mathbb{L}^2(\Omega)}.
\end{equation}
\end{lemma}

\begin{proof}
Let $\bm{u}_h \in \bm{V}_h(\textbf{curl},\Omega)$. According to Theorem \ref{regular-decomposition-of-H(curl)}, there exists $\bm{\varphi} \in \mathbb{H}^1(\Omega)$ such that
\begin{equation}\label{eq:semi-discrete-regular-decomposition}
\begin{cases}
\textbf{curl}\,(\bm{\varphi})=\textbf{curl}\,(\bm{u}_h) \in \bm{V}_h(div,\Omega)\\
\|\bm{\varphi}\|_{\mathbb{H}^1(\Omega)} \lesssim \|\textbf{curl}(\bm{u}_h)\|_{\mathbb{L}^2(\Omega)}\\
\|\bm{\varphi}\|_{\mathbb{L}^2(\Omega)} \lesssim \|\bm{u}_h\|_{\mathbb{L}^2(\Omega)}.
\end{cases}
\end{equation}
We now apply Lemma  \eqref{prelimibary-lemma-1} and obtain
$$
\textbf{curl}\,(\bm{u}_h)=\textbf{curl}\,(\bm{\varphi})=\textbf{curl}\left( \Pi_h^{\textbf{curl}} \bm{\varphi}\right),
$$
hence, 
$$\bm{u}_h - \Pi_h^{\textbf{curl}} \bm{\varphi} \in \textbf{ker}\left(\textbf{curl} \mid_{\bm{V}_h(\textbf{curl},\Omega)} \right)=\textbf{grad} \left( V_h(\textbf{grad},\Omega)\right).$$ 
Therefore, there exists $\phi_h \in V_h(\textbf{grad},\Omega)$  such that $\bm{u}_h - \Pi_h^{\textbf{curl}} \bm{\varphi}=\textbf{grad}\,(\phi_h)$,  which yields \eqref{eq:semi-discrete-decomposition}.  Estimate \eqref{eq:discrete-H-curl-estimate} follows from the second estimate in \eqref{eq:semi-discrete-regular-decomposition}.

We now show \eqref{eq:stability-of-semi-discrete-decomposition}. We write
\begin{eqnarray*}
\|\Pi_h^{\textbf{curl}} \bm{\varphi}\|_{\mathbb{L}^2(\Omega)} & \leq & \|\Pi_h^{\textbf{curl}} \bm{\varphi} -\bm{\varphi}\|_{\mathbb{L}^2(\Omega)}+\|\bm{\varphi}\|_{\mathbb{L}^2(\Omega)}\\
& \lesssim & h \|\bm{\varphi}\|_{\mathbb{H}^1(\Omega)} +  \|\bm{\varphi}\|_{\mathbb{L}^2(\Omega)}\\
& \lesssim &  h \|\textbf{curl}(\bm{u}_h)\|_{\mathbb{L}^2(\Omega)} + \|\bm{u}_h\|_{\mathbb{L}^2(\Omega)},
\end{eqnarray*}
where in the last estimate we have used first and second estimates in \eqref{eq:semi-discrete-regular-decomposition}. Moreover, using the inverse inequality 
\begin{equation}\label{eq:inverse-inequlity-curl}
\|\textbf{curl}(\bm{u}_h)\|_{\mathbb{L}^2(\Omega)} \lesssim h^{-1} \|\bm{u}_h\|_{\mathbb{L}^2(\Omega)},
\end{equation}
we get 
$$
\|\Pi_h^{\textbf{curl}} \bm{\varphi}\|_{\mathbb{L}^2(\Omega)} \lesssim \|\bm{u}_h\|_{\mathbb{L}^2(\Omega)}.
$$
On the other hand, we have 
\begin{eqnarray*}
\|\textbf{grad}\,(\phi_h)\|_{\mathbb{L}^2(\Omega)}  &=& \| \bm{u}_h - \Pi_h^{\textbf{curl}} \bm{\varphi}\|_{\mathbb{L}^2(\Omega)}\\
& \lesssim & \| \bm{u}_h \|_{\mathbb{L}^2(\Omega)} + \| \Pi_h^{\textbf{curl}} \bm{\varphi}\|_{\mathbb{L}^2(\Omega)} \lesssim \| \bm{u}_h \|_{\mathbb{L}^2(\Omega)},
\end{eqnarray*}
and  inequality \eqref{eq:stability-of-semi-discrete-decomposition} is proved.
\end{proof}

\begin{lemma}\label{lem-stable-approximation-of-vect-grad}
Every $\bm{\varphi} \in \mathbb{H}^1(\Omega)$ admits a stable approximation $\bm{\varphi}_h \in \mathbb{V}_h(\textbf{grad},\Omega)$ satisfying
$$
h^{-1} \|\bm{\varphi} -\bm{\varphi}_h \|_{\mathbb{L}^2(\Omega)} + \|\bm{\varphi}_h\|_{\mathbb{H}^1(\Omega)} \lesssim \|\bm{\varphi}\|_{\mathbb{H}^1(\Omega)}.
$$  
\end{lemma}

\begin{proof}
Let $\bm{\varphi}:=\left(\varphi^1,\varphi^2,\varphi^3\right) \in \mathbb{H}^1(\Omega)$ and define 
$$\bm{\varphi}_h= \left(\Pi_h^{\textbf{grad}} \varphi^1,\Pi_h^{\textbf{grad}} \varphi^2,\Pi_h^{\textbf{grad}} \varphi^3\right) \in \mathbb{V}_h(\textbf{grad},\Omega).$$
According to Theorem \cite[Theorem 5.3]{buffa2011isogeometric}, we have:
$$
\|\varphi^k - \Pi_h^{\textbf{grad}} \varphi^k\|_{L^2(\Omega)} \lesssim  h \|\varphi^k\|_{H^1(\Omega)}, \quad k=1,2,3 
$$
and
$$
\|\varphi^k - \Pi_h^{\textbf{grad}} \varphi^k\|_{H^1(\Omega)} \lesssim \|\varphi^k\|_{H^1(\Omega)}, \quad k=1,2,3. 
$$
Using these last two estimates, we get 
\begin{eqnarray}\label{eq:proof-stable-approximation-of-vect-grad-1}
\|\bm{\varphi} -\bm{\varphi}_h \|_{\mathbb{L}^2(\Omega)}^2 &=& \|\varphi^1 - \Pi_h^{\textbf{grad}} \varphi^1\|_{L^2(\Omega)}^2+\|\varphi^2 - \Pi_h^{\textbf{grad}} \varphi^2\|_{L^2(\Omega)}^2+\|\varphi^3 - \Pi_h^{\textbf{grad}} \varphi^3\|_{L^2(\Omega)}^2 \nonumber \\
& \lesssim & h^2 \left( \|\varphi^1\|_{H^1(\Omega)}^2 + \|\varphi^2\|_{H^1(\Omega)}^2 + \|\varphi^3\|_{H^1(\Omega)}^2 \right)= h^2 \|\bm{\varphi}\|_{\mathbb{H}^1(\Omega)}^2,
\end{eqnarray}
and
\begin{eqnarray*}
\|\bm{\varphi} -\bm{\varphi}_h \|_{\mathbb{H}^1(\Omega)}^2 &=& \|\varphi^1 - \Pi_h^{\textbf{grad}} \varphi^1\|_{H^1(\Omega)}^2+\|\varphi^2 - \Pi_h^{\textbf{grad}} \varphi^2\|_{H^1(\Omega)}^2+\|\varphi^3 - \Pi_h^{\textbf{grad}} \varphi^3\|_{H^1(\Omega)}^2\\
& \lesssim &  \|\varphi^1\|_{H^1(\Omega)}^2 + \|\varphi^2\|_{H^1(\Omega)}^2 + \|\varphi^3\|_{H^1(\Omega)}^2= \|\bm{\varphi}\|_{\mathbb{H}^1(\Omega)}^2,
\end{eqnarray*}
from which we deduce that 
\begin{equation}\label{eq:proof-stable-approximation-of-vect-grad-2}
\|\bm{\varphi}_h\|_{\mathbb{H}^1(\Omega)} \leq \|\bm{\varphi} -\bm{\varphi}_h \|_{\mathbb{H}^1(\Omega)} + \|\bm{\varphi}\|_{\mathbb{H}^1(\Omega)} \lesssim  \|\bm{\varphi}\|_{\mathbb{H}^1(\Omega)}.
\end{equation}
Combining \eqref{eq:proof-stable-approximation-of-vect-grad-1} and \eqref{eq:proof-stable-approximation-of-vect-grad-2} we conclude the proof.
\end{proof}

We have the following regular discrete decomposition.

\begin{proposition}[Hitpmair-Xu decomposition]\label{prop:Hitpmair-Xu decomposition}
Every $\bm{u}_h \in \bm{V}_h(\textbf{curl},\Omega)$ has a decomposition 
\begin{equation}\label{eq:Hitpmair-Xu-decomposition}
\bm{u}_h=\bm{w}_h+\Pi_h^{\textbf{curl}} \bm{\varphi}_h + \textbf{grad}\,(\phi_h),
\end{equation}
where $\bm{w}_h \in \bm{V}_h(\textbf{curl},\Omega)$, $\bm{\varphi}_h \in  \mathbb{V}_h(\textbf{grad},\Omega)$ and $\phi_h \in V_h(\textbf{grad},\Omega)$ with estimate
\begin{equation}\label{eq:Hitpmair-Xu-decomposition-estimate}
(h^{-2} + \mu) \, \|\bm{w}_h\|_{\mathbb{L}^2(\Omega)}^2 + \|\bm{\varphi}_h\|_{\mathbb{H}^1(\Omega)}^2 + \mu \|\bm{\varphi}_h\|_{\mathbb{L}^2(\Omega)}^2 + \mu \|\textbf{grad}\,(\phi_h)\|_{\mathbb{L}^2(\Omega)}^2 \lesssim \|\bm{u}_h\|_{A_\mu}^2,
\end{equation}
with notation
$$
\|\bm{u}_h\|_{A_\mu}^2= \|\textbf{curl} \,(\bm{u}_h)\|_{\mathbb{L}^2(\Omega)}^2 + \mu \|\bm{u}_h\|_{\mathbb{L}^2(\Omega)}^2.
$$
\end{proposition} 

\begin{proof}
Let $\bm{u}_h \in \bm{V}_h(\textbf{curl},\Omega)$. Using lemma \eqref{lem-semi-discrete-decomposition} we can find 
$\bm{\varphi} \in \mathbb{H}^1(\Omega)$ and $\phi_h \in V_h(\textbf{grad},\Omega)$ with the properties
\begin{equation}\label{pr-eq:semi-discrete-decomposition}
\begin{cases}
\bm{u}_h = \Pi_h^{\textbf{curl}} \bm{\varphi} + \textbf{grad}\,(\phi_h)\\
\|\bm{\varphi}\|_{\mathbb{H}^1(\Omega)} \lesssim \|\textbf{curl} \,(\bm{u}_h)\|_{\mathbb{L}^2(\Omega)}\\
\|\Pi_h^{\textbf{curl}} \bm{\varphi}\|_{\mathbb{L}^2(\Omega)} + \|\textbf{grad}\,(\phi_h)\|_{\mathbb{L}^2(\Omega)}\lesssim \|\bm{u}_h\|_{\mathbb{L}^2(\Omega)},
\end{cases}
\end{equation}
and let $\bm{\varphi}_h \in \mathbb{V}_h(\textbf{grad},\Omega)$ the stable approximation given by Lemma \ref{lem-stable-approximation-of-vect-grad}. Let us define
$$
\bm{w}_h=\Pi_h^{\textbf{curl}} (\bm{\varphi}-\bm{\varphi}_h).
$$
In this way, using the decomposition in \eqref{pr-eq:semi-discrete-decomposition}, we obtain
\begin{eqnarray*}
\bm{u}_h = \Pi_h^{\textbf{curl}} \bm{\varphi} + \textbf{grad}\,(\phi_h)&=&\Pi_h^{\textbf{curl}} (\bm{\varphi}-\bm{\varphi}_h)+ \Pi_h^{\textbf{curl}} \bm{\varphi}_h+ \textbf{grad}\,(\phi_h)\\
&=& \Pi_h^{\textbf{curl}} \bm{w}_h+ \Pi_h^{\textbf{curl}} \bm{\varphi}_h+ \textbf{grad}\,(\phi_h),
\end{eqnarray*}
and decomposition \eqref{eq:Hitpmair-Xu-decomposition} is proved. In order to show \eqref{eq:Hitpmair-Xu-decomposition-estimate}, we need to perform careful estimates. Indeed we have
\begin{eqnarray}\label{pr-eq:semi-discrete-decomposition-1}
h^{-1} \|\bm{w}_h\|_{\mathbb{L}^2(\Omega)} &=& h^{-1} \|\Pi_h^{\textbf{curl}} (\bm{\varphi}-\bm{\varphi}_h)\|_{\mathbb{L}^2(\Omega)}  \nonumber \\ 
& \lesssim & h^{-1}  \|\bm{\varphi}-\bm{\varphi}_h)\|_{\mathbb{L}^2(\Omega)}
 \lesssim  \|\bm{\varphi}\|_{\mathbb{H}^1(\Omega)} \lesssim \|\textbf{curl} \,(\bm{u}_h)\|_{\mathbb{L}^2(\Omega)},
\end{eqnarray}
where in the last estimate we have used the first inequality in \eqref{pr-eq:semi-discrete-decomposition}. Moreover, using the inverse inequality \eqref{eq:inverse-inequlity-curl} we get
\begin{equation}
\|\bm{w}_h\|_{\mathbb{L}^2(\Omega)} \lesssim h \|\textbf{curl} \,(\bm{u}_h)\|_{\mathbb{L}^2(\Omega)} \lesssim   \|\bm{u}_h\|_{\mathbb{L}^2(\Omega)}.
\end{equation}
Concerning the component $\bm{\varphi}_h$, we use first inequality in \eqref{pr-eq:semi-discrete-decomposition} to obtain
\begin{equation}
\|\bm{\varphi}_h\|_{\mathbb{H}^1(\Omega)} \lesssim \|\bm{\varphi}\|_{\mathbb{H}^1(\Omega)} \lesssim \|\textbf{curl} \,(\bm{u}_h)\|_{\mathbb{L}^2(\Omega)},
\end{equation}
and
\begin{equation}\label{pr-eq:semi-discrete-decomposition-4}
\|\bm{\varphi}_h\|_{\mathbb{L}^2(\Omega)} \leq \|\bm{\varphi}_h\|_{\mathbb{H}^1(\Omega)}  \lesssim  \|\bm{\varphi}\|_{\mathbb{H}^1(\Omega)}  \lesssim \|\textbf{curl} \,(\bm{u}_h)\|_{\mathbb{L}^2(\Omega)}.
\end{equation}
Combining \eqref{pr-eq:semi-discrete-decomposition-1}--\eqref{pr-eq:semi-discrete-decomposition-4} together with second estimate in \eqref{pr-eq:semi-discrete-decomposition}, and using the fact that $0 < \mu \leq 1$, we obtain the desired estimate  \eqref{eq:Hitpmair-Xu-decomposition-estimate}. This complete the proof of the proposition. 
\end{proof}

This last proposition leads to the following stable decomposition of $\bm{V}_h(\textbf{curl},\Omega)$, which is a discrete version of Theorem \ref{regular-decomposition-of-H(curl)}.

\begin{theorem}[Stable Hitpmair-Xu decomposition]\label{thm:stable-Hitpmair-Xu-decomposition}
For each $\bm{u}_h \in \bm{V}_h(\textbf{curl},\Omega)$, there exit $\bm{w}_h \in \bm{V}_h(\textbf{curl},\Omega)$, $\bm{\varphi}_h \in  \mathbb{V}_h(\textbf{grad},\Omega)$ and $\phi_h \in V_h(\textbf{grad},\Omega)$ such that   
\begin{equation}\label{eq:stable-Hitpmair-Xu-decomposition}
\bm{u}_h=\bm{w}_h+\Pi_h^{\textbf{curl}} \bm{\varphi}_h + \textbf{grad}\,(\phi_h),
\end{equation}
and 
\begin{equation}\label{eq:stable-Hitpmair-Xu-decomposition-estimate}
\|\bm{w}_h\|_{A_\mu}^2 + \|\bm{\varphi}_h\|_{\mathbb{H}^1(\Omega)}^2 + \mu \|\bm{\varphi}_h\|_{\mathbb{L}^2(\Omega)}^2 + \|\textbf{grad}\,(\phi_h)\|_{A_\mu}^2 \lesssim \|\bm{u}_h\|_{A_\mu}^2.
\end{equation}
\end{theorem}

\begin{proof}
The components $\bm{w}_h$, $\bm{\varphi}_h$ and $\phi_h$ are chosen as in Proposition \ref{prop:Hitpmair-Xu decomposition}. Hence, by remaking that 
$$
\|\textbf{grad}\,(\phi_h)\|_{A_\mu}^2= \|\textbf{curl} \,\left( \textbf{grad}\,(\phi_h) \right)\|_{\mathbb{L}^2(\Omega)}^2 + \mu \, \|\textbf{grad}\,(\phi_h)\|_{\mathbb{L}^2(\Omega)}^2 =\mu \|\textbf{grad}\,(\phi_h)\|_{\mathbb{L}^2(\Omega)}^2,
$$
we need only to estimate the first term in \eqref{eq:Hitpmair-Xu-decomposition-estimate}. For which, we have 
\begin{eqnarray*}
(h^{-2} + \mu) \, \|\bm{w}_h\|_{\mathbb{L}^2(\Omega)}^2 &=& h^{-2}\, \|\bm{w}_h\|_{\mathbb{L}^2(\Omega)}^2+ \mu \, \|\bm{w}_h\|_{\mathbb{L}^2(\Omega)}^2\\
& \gtrsim & \|\textbf{curl} \,(\bm{w}_h)\|_{\mathbb{L}^2(\Omega)}^2 + \mu \|\bm{w}_h\|_{\mathbb{L}^2(\Omega)}^2 = \|\bm{w}_h\|_{A_\mu}^2.
\end{eqnarray*}
This completes the proof.
\end{proof}

%{\color{red} \subsection{Discrete Decompositions with Mapping (To be discussed)}\label{subsec:asp-with-mapping}}

\subsection{Auxiliary Space Preconditioners}\label{subsec:construction-of-ASP}
Now we have all the ingredients to apply the abstract ASP theory of Section \ref{subsec:Auxiliary-Space-Preconditioning-Method}. Indeed, armed with the above stable discrete decomposition result, using the notations of Section \ref{subsec:Auxiliary-Space-Preconditioning-Method}, we consider $V=\bm{V}_h(\textbf{curl},\Omega)$ equipped with the bilinear form $a$ related to equation \eqref{eq:curl-variational-problem}, namely $a(\bm{w}_h,\widetilde{\bm{w}_h})= \left(\textbf{curl}\,(\bm{w}_h),\textbf{curl}\,(\widetilde{\bm{w}_h})\right)_{\mathbb{L}^2(\Omega)} + \mu (\bm{w}_h,\widetilde{\bm{w}_h})_{\mathbb{L}^2(\Omega)}$ for $\bm{w}_h, \widetilde{\bm{w}_h} \in V$, and auxiliary spaces $W_1= \mathbb{V}_h(\textbf{curl},\Omega)$ and $W_2=V_h(\textbf{grad},\Omega)$ equipped with the following inner products
\begin{equation*}
a_1(\bm{\varphi}_h,\widetilde{\bm{\varphi}_h})= (\bm{\varphi}_h,\widetilde{\bm{\varphi}_h})_{\mathbb{H}^1(\Omega)} + \mu (\bm{\varphi}_h,\widetilde{\bm{\varphi}_h})_{\mathbb{L}^2(\Omega)}, \quad \bm{\varphi}_h,\widetilde{\bm{\varphi}_h} \in W_1,
\end{equation*}
and
\begin{equation*}
a_2(\phi_h,\widetilde{\phi_h})= \mu \, (\textbf{grad}\,(\phi_h),\textbf{grad}\,(\widetilde{\phi_h}))_{\mathbb{L}^2(\Omega)},  \quad \phi_h,\widetilde{\phi_h} \in W_2,
\end{equation*}
respectively. The corresponding transfer operators are $\pi_1 = \Pi_h^{\textbf{curl}}\,:\, W_1 \longrightarrow V$ and $\pi_2=\textbf{grad}\,:\, W_2 \longrightarrow V$ and the smoother is simply the inner product $a$. 

Before we verify the validity of assumptions of Theorem \ref{th:ASP-lemma}, we transition to a matrix notation. So we write $C_h^1$ for the matrix related to the restriction of $\mathbb{H}^1(\Omega)$ inner product to $\mathbb{V}_h(\textbf{grad},\Omega)$, that is the matrix representation of the bilinear form 
\begin{equation*}
(\bm{\varphi}_h,\widetilde{\bm{\varphi}_h}) \in \mathbb{V}_h(\textbf{grad},\Omega) \times \mathbb{V}_h(\textbf{grad},\Omega)  \mapsto (\bm{\varphi}_h,\widetilde{\bm{\varphi}_h})_{\mathbb{H}^1(\Omega)}.
\end{equation*}
Similarly, we write $C_h^2$ for the matrix related to the restriction of the $\mathbb{L}^2(\Omega)$ inner product to $\mathbb{V}_h(\textbf{grad},\Omega)$. Let $L_h$ be the discrete Laplacian matrix, namely the matrix related to the mapping  
\begin{equation*}
(\bm{\phi}_h,\widetilde{\bm{\phi}_h}) \in V_h(\textbf{grad},\Omega) \times V_h(\textbf{grad},\Omega)  \longmapsto \left(\textbf{grad}\,(\bm{\phi}_h),\textbf{grad}\,(\widetilde{\bm{\phi}_h})\right)_{\mathbb{L}^2(\Omega)}.
\end{equation*}
We also write $P_h^{\textbf{curl}}$ and $G_h$  for the matrices related to the transform operators $\pi_1$ and $\pi_2$ respectively while $R_h$ further stands for the matrix related to the smoother. With these notations, a simple computation shows that ASP preconditioner  for problem \eqref{eq:curl-variational-problem} reads 
\begin{equation*}
B_h=R_h^{-1} + P_h^{\textbf{curl}}  \left(C_h^1+\mu C_h^2 \right)^{-1} \left( P_h^{\textbf{curl}}\right)^T + \mu^{-1} G_h L_h^{-1} \left(G_h\right)^T. 
\end{equation*}

A direct consequence of Theorem \ref{thm:stable-Hitpmair-Xu-decomposition} and Theorem \ref{th:ASP-lemma} the following fundamental result which show the mesh-independent of the preconditioner.
\begin{theorem}
For $0 < \mu \leq 1$, the spectral condition number $\kappa \left( B_h A_h\right)$ is bounded, with respect to $h$.
\end{theorem}





















