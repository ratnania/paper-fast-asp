%\section{Global approximation}
%\subsection{Splines Interpolation}
%Given the set of interpolations points $X:=\{ x_0, \cdots,x_n\}$ and data $Y:=\{ y_0, \cdots,y_n\}$, we aim to find a spline $s$ such that $y_i = s(x_i)$ . The Spline interpolation problems is
%\begin{definition}[Spline interpolation]
%  Find a spline $s := \sum\limits_{j=0}^n c_j N_j^p \in \mathcal{S}_p(T)$ such that
%  \begin{align}
%    s(x_i) = y_{i} \quad \quad 0 \leq i \leq n
%%    \label{}
%  \end{align}
%\end{definition}
%In a matrix form, the Spline interpolation problem writes
%\begin{align}
%  M c = y
%  \label{eq:interpolation-linsys}
%\end{align}
%where $c$ is the unkown vector of the spline coefficients, 
%\begin{align*}
%  c =  
%  \begin{pmatrix}
%    c_0\\
%    c_1\\
%    \vdots    \\ 
%    c_n\\
%  \end{pmatrix}
%\end{align*}
%$M$ is the \textbf{collocation matrix} given by
%\begin{align}
%  M = 
%  \begin{pmatrix}
%    N_0^p(x_0) & \ldots & N_n^p(x_0) \\
%    N_0^p(x_1) & \ldots & N_n^p(x_1) \\
%    \vdots     & \ldots & \vdots \\
%    N_0^p(x_n) & \ldots & N_n^p(x_n) \\
%  \end{pmatrix}
%  \label{eq:collocation-matrix}
%\end{align}
%while $y$ is the given data
%\begin{align*}
%  y =  
%  \begin{pmatrix}
%    y_0\\
%    y_1\\
%    \vdots    \\ 
%    y_n\\
%  \end{pmatrix}
%\end{align*}
%Notice that when interpolating a function $f$, the given data will be $y_i := f(x_i)$.
%\\
%\noindent
%In general the linear system given by the Eq. \ref{eq:interpolation-linsys} is not always solvable. The following result by Whitney-Schoenberg gives a necessary and sufficient condition to ensure that the interpolation problem has a unique solution. 
%\begin{theorem}
%  The collocation matrix is nonsingular if and only if the diagonal elements are positive, \textit{i.e.}
%  \begin{align}
%    N_i^p(x_i) > 0 \quad \forall i \in \{ 0, \ldots, n \}
%    \label{eq:whitney-schoenberg-condition}
%  \end{align}
%  This condition is also equivalent to 
%  \begin{align}
%    t_i < x_i < t_{i+p+1} \quad \forall i \in \{ 0, \ldots, n \}
%    \label{eq:whitney-schoenberg-condition-knots}
%  \end{align}
%  \label{thm:whitney-schoenberg}
%\end{theorem}
\subsection{Splines Histopolation}
In this subsection, we present the histopolation problem and its matrix form.
In opposition to the interpolation problem, where we preserve the values of a function on a given nodes, another interesting way to approximate a function,  is to preserve the integrals between given points, rather than the value of the function on these points. Given the set of interpolations points $X:=\{ x_0, \cdots,x_{n} \}$ and a continuous function $f$, the histopolation problem writes 
\begin{definition}[Spline histopolation]
  Find a spline $s := \sum\limits_{j=1}^{n} c_j N_j^p \in \mathcal{S}_p(T)$ such that
  \begin{align}
    \int_{x_i}^{x_{i+1}} s ~dx = \int_{x_i}^{x_{i+1}} f ~dx \quad \quad 1 \leq i \leq n
%    \label{}
  \end{align}
\end{definition}
\noindent
In a matrix form, the Spline histopolation problem writes $ H c = y $ where $H$ is the \textbf{histopolation matrix}, $c$ is the unkown vector of the spline coefficients and $y$ is the given data, are given by
\begin{align*}
  c =  
  \begin{bmatrix}
    c_1\\
    c_2\\
    \vdots    \\ 
    c_n\\
  \end{bmatrix}
  \quad
  H = 
  \begin{bmatrix}
    \int_{x_0}^{x_{1}}N_1^p ~dx   & \ldots & \int_{x_0}^{x_{1}}N_n^p ~dx   \\
    \int_{x_1}^{x_{2}}N_1^p ~dx   & \ldots & \int_{x_1}^{x_{2}}N_n^p ~dx   \\
    \vdots                        & \ldots &                        \vdots \\
    \int_{x_n}^{x_{n+1}}N_1^p ~dx & \ldots & \int_{x_n}^{x_{n+1}}N_n^p ~dx \\
  \end{bmatrix}
  \quad
  y =  
  \begin{bmatrix}
    \int_{x_0}^{x_{1}}f ~dx\\
    \int_{x_1}^{x_{2}}f ~dx\\
    \vdots    \\ 
    \int_{x_n}^{x_{n+1}}f ~dx\\
  \end{bmatrix}
%  \label{eq:histopolation-matrix}
\end{align*}
We recall that the histopolation matrix $H$ is non singular iff  $t_i < x_i < t_{i+p+1} \quad \forall i \in \{ 0, \ldots, n \}$ \todo{add citation}.
%\begin{theorem}
%  The histopolation matrix is nonsingular if and only if 
%  \begin{align}
%    t_i < x_i < t_{i+p+1} \quad \forall i \in \{ 0, \ldots, n \}
%    \label{eq:whitney-schoenberg-condition-histo-knots}
%  \end{align}
%  \label{thm:whitney-schoenberg-histo}
%\end{theorem}
%\begin{proof}
%  TODO
%\end{proof}

\paragraph{Histopolation using M-Splines} \mbox{}\\
Rather than using the B-Splines for the histopolation problem, one can use the M-Splines. In this case, the histopolation matrix can be computed easily using the following result.
\begin{proposition}
  For every $0 \le i \le  n$ and $0 \le j \le n$, we have
  \begin{align}
    \int_{x_i}^{x_{i+1}} M_j^p(t) ~dt = \sum_{k=0}^{j-1} \left( N_k^p(x_i) - N_k^p(x_{i+1}) \right)  
%    \label{}
  \end{align}
%  \label{}
\end{proposition}
\begin{proof}
Integrating the relation $\frac{d}{dt}N_k^p(t)=M_k^{p}(t)-M_{k+1}^{p}(t)$ on the interval $[x_i, x_{i+1}]$, we have
\begin{align*}
  N_k^p(x_{i+1}) - N_k^p(x_i) = \int_{x_i}^{x_{i+1}} \left( M_k^{p}(t)-M_{k+1}^{p}(t) \right) ~dt 
%  \label{}
\end{align*}
summing the last equation for $k=0$ to $k=j-1$, we get
\begin{align*}
  \sum_{k=0}^{j-1} \left( N_k^p(x_{i+1}) - N_k^p(x_i) \right) 
  &= \int_{x_i}^{x_{i+1}} \sum_{k=0}^{j-1} \left( M_k^{p}(t)-M_{k+1}^{p}(t) \right) ~dt 
  \\
  &= \int_{x_i}^{x_{i+1}} \left( M_0^{p}(t)-M_{j}^{p}(t) \right) ~dt 
%  \label{}
\end{align*}
hence,
\begin{align*}
  \int_{x_i}^{x_{i+1}} M_{j}^{p}(t) ~dt = \sum_{k=0}^{j-1} \left( N_k^p(x_{i}) - N_k^p(x_{i+1}) \right) 
%  \label{}
\end{align*}
\end{proof}
%\begin{remark}
%  The last result gives an optimized implementation for the assembly of the histopolation matrix, since the right hand side term can be computed by accumulating the summation for each $j$.
%\end{remark}

%\paragraph{Commuting diagram} \mbox{}\\
%Let us define the following discrete spline spaces 
%\begin{align}
%  V_0 := \mbox{span}\left( N_j^p, 0 \leq j \leq n  \right) 
%  \\
%  V_1 := \mbox{span}\left( M_j^{p-1}, 0 \leq j \leq n \right) 
%%  \label{}
%\end{align}
%We also define the operators $\pi_0$ and $\pi_1$ as the interpolation and histopolation operators on $V_0$ and $V_1$ respectively.
%We have the following result
%%
%\begin{lemma}
%  $\forall u \in H^s(\Omega)$, we have $~~ \diff \pi_0(u) = \pi_1 \left( \diff u \right) $
%%  \label{}
%\end{lemma}
\noindent
A special case, which is of interest for our Auxiliary Space Preconditioning method, is when the function lives in $\mathcal{S}_p(T)$ and the histopolation is done on $\mathcal{S}_{p-1}(T)$ using M-Splines. More precisely, we are interested in the case where Homogeneous Dirichlet boundary conditions are imposed and the interpolating degrees of freedom are removed. 
\noindent
Let $u := \sum\limits_{1 \le j \le n-1} u_j \Njone$, we have
\begin{align}
  \int_{x_i}^{x_{i+1}} u ~dx = \sum\limits_{1 \le j \le n} u_j \int_{x_i}^{x_{i+1}} \Njone ~dx \quad \quad 1 \leq i \leq n-1
\end{align}
which can be written in a matrix form as
\begin{align*}
  \begin{bmatrix}
    \int_{x_0}^{x_{1}}u ~dx\\
    \int_{x_1}^{x_{2}}u ~dx\\
    \vdots    \\ 
    \int_{x_n}^{x_{n+1}}u ~dx\\
  \end{bmatrix}
  &=
  \begin{bmatrix}
    \int_{x_0}^{x_{1}}N_1^p ~dx   & \ldots & \int_{x_0}^{x_{1}}N_{n-1}^p ~dx   \\
    \int_{x_1}^{x_{2}}N_1^p ~dx   & \ldots & \int_{x_1}^{x_{2}}N_{n-1}^p ~dx   \\
    \vdots                        & \ldots &                        \vdots \\
    \int_{x_n}^{x_{n+1}}N_1^p ~dx & \ldots & \int_{x_n}^{x_{n+1}}N_{n-1}^p ~dx \\
  \end{bmatrix}
  \begin{bmatrix}
    u_1\\
    u_2\\
    \vdots\\ 
    u_{n-1}\\
  \end{bmatrix}
  \\
  &= 
  \begin{bmatrix}
    \int_{x_0}^{x_{1}}M_1^{p-1} ~dx   & \ldots & \int_{x_0}^{x_{1}}M_{n-1}^{p-1} ~dx   \\
    \int_{x_1}^{x_{2}}M_1^{p-1} ~dx   & \ldots & \int_{x_1}^{x_{2}}M_{n-1}^{p-1} ~dx   \\
    \vdots                        & \ldots &                        \vdots \\
    \int_{x_n}^{x_{n+1}}M_1^{p-1} ~dx & \ldots & \int_{x_n}^{x_{n+1}}M_{n-1}^{p-1} ~dx \\
  \end{bmatrix}
  \begin{bmatrix}
    u_1^\star\\
    u_2^\star\\
    \vdots\\ 
    u_{n-1}^\star\\
  \end{bmatrix}
%  \label{}
\end{align*}
Finaly, we get
\begin{align}
  U^\star = \left( H^{M} \right)^{-1} H^{B} U  
%  \label{}
\end{align}
with
\begin{align}
  H^{M} &:= 
  \begin{bmatrix}
    \int_{x_0}^{x_{1}}M_1^{p-1} ~dx   & \ldots & \int_{x_0}^{x_{1}}M_{n-1}^{p-1} ~dx   \\
    \int_{x_1}^{x_{2}}M_1^{p-1} ~dx   & \ldots & \int_{x_1}^{x_{2}}M_{n-1}^{p-1} ~dx   \\
    \vdots                        & \ldots &                        \vdots \\
    \int_{x_n}^{x_{n+1}}M_1^{p-1} ~dx & \ldots & \int_{x_n}^{x_{n+1}}M_{n-1}^{p-1} ~dx \\
  \end{bmatrix}
  \\
  H^{B} &:= 
  \begin{bmatrix}
    \int_{x_0}^{x_{1}}N_1^p ~dx   & \ldots & \int_{x_0}^{x_{1}}N_{n-1}^p ~dx   \\
    \int_{x_1}^{x_{2}}N_1^p ~dx   & \ldots & \int_{x_1}^{x_{2}}N_{n-1}^p ~dx   \\
    \vdots                        & \ldots &                        \vdots \\
    \int_{x_n}^{x_{n+1}}N_1^p ~dx & \ldots & \int_{x_n}^{x_{n+1}}N_{n-1}^p ~dx \\
  \end{bmatrix}
  \label{eq:histopolation-h10}
\end{align}
