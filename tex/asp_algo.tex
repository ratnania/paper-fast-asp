\section{Auxiliary Space Preconditioning - Algorithms}

In the sequel, we present different algorithms based on our regular decomposition (Th. \ref{thm:stable-Hitpmair-Xu-decomposition}).
\\
We shall denote
\begin{description}
  \item[$\mbox{\texttt{smoother}}_1$ :] any relaxation method, such as (symmetric)-\texttt{Gauss-Seidel}, \texttt{Jacobi} or \texttt{SOR}
  \item[$\nu_1 \quad \quad \quad~$ :] number of sweeps for $\mbox{\texttt{smoother}}_1$
  \item[$\mbox{\texttt{smoother}}_2$ :] preconditioned \texttt{CG} or \texttt{GMRES} using the matrix $T_p$
  \item[$\nu_2 \quad \quad \quad~$ :] number of iterations for $\mbox{\texttt{smoother}}_2$. In practice, it is set to $\max{p_1, p_2, p_3}+1$
\end{description}

% ...
\begin{algorithm}[H]
\DontPrintSemicolon
\SetAlgoLined
\SetKwInOut{Input}{Input}\SetKwInOut{Output}{Output}
\Input{$A, B, b, u_0, \nu_1, n_{\max}$}
\Output{$u_k$}
\BlankLine

  $k \gets 0$\;
  \While{$k \leq n_{\max}$ \textbf{and} not convergence}{
    $u_k \gets \mbox{\texttt{smoother}}_1(A,b,u_k,\nu_1)$\;
    $d_k \gets b - A u_k$      \tcp*{compute the defect}
    $u_c \gets B d_k$          \tcp*{ASP correction}
    $u_k \gets u_k + u_c$      \tcp*{Update the solution}
    $k \gets k + 1$\;
  }

\caption{{\sc ASP-1}: Auxiliary Space Preconditioning for $\bm{V}_h(\textbf{curl},\Omega)$}
\end{algorithm} 
% ...

% ...
\begin{algorithm}[H]
\DontPrintSemicolon
\SetAlgoLined
\SetKwInOut{Input}{Input}\SetKwInOut{Output}{Output}
\Input{$A, B, T_p, b, u_0, \nu_1, \nu_2, n_{\max}$}
\Output{$u_k$}
\BlankLine

  $k \gets 0$\;
  \While{$k \leq n_{\max}$ \textbf{and} not convergence}{
    $u_k \gets \mbox{\texttt{smoother}}_1(A,b,\nu_1,u_k)$\;
    $u_k \gets \mbox{\texttt{smoother}}_2(A,b,\nu_2,u_k,P=M^{-1})$\;
    $d_k \gets b - A u_k$      \tcp*{compute the defect}
    $u_c \gets B d_k$          \tcp*{ASP correction}
    $u_k \gets u_k + u_c$      \tcp*{Update the solution}
    $k \gets k + 1$\;
  }

\caption{{\sc ASP-2}: Auxiliary Space Preconditioning for $\bm{V}_h(\textbf{curl},\Omega)$ with optimal dependency with respect to the Spline degree}
\end{algorithm} 
% ...
