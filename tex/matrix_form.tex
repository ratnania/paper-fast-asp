\section{Matrix form of the variational problem}
In the sequel, we derive the matrix form related to the discrete variational formulation of \eqref{eq:curl-variational-problem}. 
We shall consider a computational domain $\Omega$ as unit square $(0,1)^2$ or unit cube $(0,1)^3$ and derive the matrix form using the Kronecker Algebra. 
Before expliciting the matrix forms in 2D and 3D, we start by introducing some 1D matrices that we will need.
\begin{align}
  \begin{cases}
    \left( M_{s} \right)_{i_s, j_s} &= \int_{0}^1 \Nis \Njs \dd x_s
    \\
    \left( K_{s} \right)_{i_s, j_s} &= \int_{0}^1 \Nis^\prime \Njs^\prime \dd x_s
    \\
    \left( D_{s} \right)_{i_s, j_s} &= \int_{0}^1 \Mis \Mjs \dd x_s
    \\
    \left( R_{s} \right)_{i_s, j_s} &= \int_{0}^1 \Nis^\prime \Mjs \dd x_s
  \end{cases}
\end{align}

\subsection{2D case}
We consider the discrete variational formulation of \eqref{eq:curl-variational-problem}. 
For the sake of simplicity we shall introduce the scalar functions
\begin{align*}
  \Psi^1_{\jj} = \Mjone \Njtwo 
  \\
  \Psi^2_{\jj} = \Njone \Mjtwo  
\end{align*}
we also define the vectors $\ee_1 = \begin{bmatrix} 1 \\ 0 \end{bmatrix}$ and $\ee_2 = \begin{bmatrix} 0 \\ 1 \end{bmatrix}$. Therefor, the expression of $\uu_h \in \Vcurl$ becomes
\begin{align*}
  \uu_h = \sum\limits_{\jj} \left( 
    u_{\jj}^1 \Psi^1_{\jj} \ee_1 
  + u_{\jj}^2 \Psi^2_{\jj} \ee_2 
  \right)
\end{align*}
we find that $A$ is a symmetric $2\times 2$ block matrix of the form
\begin{align}
  A = 
  \begin{bmatrix}
    A_{11}   & A_{12} \\
    A_{12}^T & A_{22} \\
  \end{bmatrix}
%  \label{}
\end{align}
where
\begin{align*}
  {A_{11}}_{\ii, \jj} &=   \int_{\Omega} 
         \partial_{x_2} \Psi^{1}_{\jj} \partial_{x_2} \Psi^{1}_{\ii}
        \dd \mathbf{x} 
                       +   \int_{\Omega} 
                       \tau \Psi^{1}_{\jj} \Psi^{1}_{\ii} 
        \dd \mathbf{x} 
  \\
  {A_{12}}_{\ii, \jj} &= - \int_{\Omega} 
          \partial_{x_2} \Psi^{1}_{\jj} \partial_{x_1} \Psi^{2}_{\ii}
        \dd \mathbf{x} 
  \\
  {A_{22}}_{\ii, \jj} &=   \int_{\Omega} 
        \partial_{x_1} \Psi^{2}_{\jj} ~ \partial_{x_1} \Psi^{2}_{\ii}
        \dd \mathbf{x} 
                       +   \int_{\Omega} 
                       \tau \Psi^{2}_{\jj} \Psi^{2}_{\ii} 
        \dd \mathbf{x} 
\end{align*}
For the right hand side, the entries associated to each component of the vector $\ff$ are given by
\begin{align*}
    F_{1, i} & = \int_{\Omega} \ff_1 \Psi^1_{i} \dd \mathbf{x} \\
    F_{2, i} & = \int_{\Omega} \ff_2 \Psi^2_{i} \dd \mathbf{x} \\
\end{align*}
Hence, we have,
\begin{align*}
  \begin{cases}
    {A_{11}}_{\ii, \jj} &= \left( D_1 \otimes K_2 \right)_{\ii \jj} 
                         + \tau \left( D_1 \otimes M_2 \right)_{\ii \jj} 
    \\
    {A_{12}}_{\ii, \jj} &= - \left( R_1 \otimes R_2^T \right)_{\ii \jj} 
    \\
    {A_{22}}_{\ii, \jj} &= \left( K_1 \otimes D_2 \right)_{\ii \jj} 
                         + \tau \left( M_1 \otimes D_2 \right)_{\ii \jj} 
  \end{cases}
\end{align*}
Therefor, we have the following matrix form 
\begin{align}
  \mathcal{A}^\tau = \mathcal{A}^0 + \tau \mathcal{M}
%  \label{}
\end{align}
where
\begin{align}
  \mathcal{A}^0 = 
  \begin{bmatrix}
        D_1 \otimes K_2  & - R_1 \otimes R_2^T  \\
    - R_1^T \otimes R_2  &   K_1 \otimes D_2 
  \end{bmatrix}
  \label{eq:matrix-A0-2d}
\end{align}
and
\begin{align}
  \mathcal{M} = 
  \begin{bmatrix}
    D_1 \otimes M_2   & 0  \\
    0 & M_1 \otimes D_2   
  \end{bmatrix}
  \label{eq:matrix-M-2d}
\end{align}

\subsection{3D case}
We consider the discrete variational formulation of \eqref{eq:curl-variational-problem}. 
For the sake of simplicity we shall introduce the scalar functions
\begin{align*}
  \Psi^1_{\jj} = \Mjone \Njtwo \Njtre 
  \\
  \Psi^2_{\jj} = \Njone \Mjtwo \Njtre  
  \\
  \Psi^3_{\jj} = \Njone \Njtwo \Mjtre  
\end{align*}
we also define the vectors $\ee_1 = \begin{bmatrix} 1 \\ 0 \\ 0 \end{bmatrix}$, $\ee_2 = \begin{bmatrix} 0 \\ 1 \\ 0 \end{bmatrix}$ and $\ee_3 = \begin{bmatrix} 0 \\ 0 \\ 1 \end{bmatrix}$. Therefor, the expression of $\uu_h \in \Vcurl$ becomes
\begin{align*}
  \uu_h = \sum\limits_{\jj} \left( 
    u_{\jj}^1 \Psi^1_{\jj} \ee_1 
  + u_{\jj}^2 \Psi^2_{\jj} \ee_2 
  + u_{\jj}^3 \Psi^3_{\jj} \ee_3 
  \right)
\end{align*}
On the other hand, we have,
\begin{align*}
 \Curl \Psi^{1}_{\jj} \ee_{1} &= 
   \partial_{x_3} \Psi^{1}_{\jj} \ee_2 
 - \partial_{x_2} \Psi^{1}_{\jj} \ee_3
 \\
 \Curl \Psi^{2}_{\jj} \ee_{2} &= 
 - \partial_{x_3} \Psi^{2}_{\jj} \ee_1
 + \partial_{x_1} \Psi^{2}_{\jj} \ee_3
 \\
 \Curl \Psi^{3}_{\jj} \ee_{3} &= 
   \partial_{x_2} \Psi^{3}_{\jj} \ee_1 
 - \partial_{x_1} \Psi^{3}_{\jj} \ee_2
\end{align*}
Because $\ee_i \cdot \ee_j = \delta_{ij}$, we get, 
\begin{align*}
 \Curl \Psi^{1}_{\jj} \ee_{1} \cdot \Curl \Psi^{1}_{\ii} \ee_{1} &=  
    \partial_{x_3} \Psi^{1}_{\jj} ~ \partial_{x_3} \Psi^{1}_{\ii} 
  + \partial_{x_2} \Psi^{1}_{\jj} ~ \partial_{x_2} \Psi^{1}_{\ii}
 \\
 \Curl \Psi^{1}_{\jj} \ee_{1} \cdot \Curl \Psi^{2}_{\ii} \ee_{2} &= 
 - \partial_{x_2} \Psi^{1}_{\jj} ~ \partial_{x_1} \Psi^{2}_{\ii}
 \\
 \Curl \Psi^{1}_{\jj} \ee_{1} \cdot \Curl \Psi^{3}_{\ii} \ee_{3} &=  
 - \partial_{x_3} \Psi^{1}_{\jj} ~ \partial_{x_1} \Psi^{3}_{\ii}
 \\
% \Curl \Psi^{2}_{\jj} \ee_{2} \cdot \Curl \Psi^{1}_{\ii} \ee_{1} &=  
% - \partial_{x_1} \Psi^{2}_{\jj} ~ \partial_{x_2} \Psi^{1}_{\jj}
% \\
 \Curl \Psi^{2}_{\jj} \ee_{2} \cdot \Curl \Psi^{2}_{\ii} \ee_{2} &= 
   \partial_{x_3} \Psi^{2}_{\jj} ~ \partial_{x_3} \Psi^{2}_{\ii} 
 + \partial_{x_1} \Psi^{2}_{\jj} ~ \partial_{x_1} \Psi^{2}_{\ii}
 \\
 \Curl \Psi^{2}_{\jj} \ee_{2} \cdot \Curl \Psi^{3}_{\ii} \ee_{3} &=  
 - \partial_{x_3} \Psi^{2}_{\jj} ~  \partial_{x_2} \Psi^{3}_{\ii} 
 \\
% \Curl \Psi^{3}_{\jj} \ee_{3} \cdot \Curl \Psi^{1}_{\ii} \ee_{1} &=  
% - \partial_{x_1} \Psi^{3}_{\jj} ~ \partial_{x_3} \Psi^{1}_{\ii} 
% \\
% \Curl \Psi^{3}_{\jj} \ee_{3} \cdot \Curl \Psi^{2}_{\ii} \ee_{2} &= 
% - \partial_{x_2} \Psi^{3}_{\jj} ~ \partial_{x_3} \Psi^{2}_{\ii} 
% \\
 \Curl \Psi^{3}_{\jj} \ee_{3} \cdot \Curl \Psi^{3}_{\ii} \ee_{3} &=  
   \partial_{x_2} \Psi^{3}_{\jj} ~ \partial_{x_2} \Psi^{3}_{\ii} 
 + \partial_{x_1} \Psi^{3}_{\jj} ~ \partial_{x_1} \Psi^{3}_{\ii}
\end{align*}
we find that $A$ is a symmetric $3\times 3$ block matrix of the form
\begin{align}
  A = 
  \begin{bmatrix}
    A_{11}   & A_{12}   &  A_{13} \\
    A_{12}^T & A_{22}   &  A_{23} \\
    A_{13}^T & A_{23}^T &  A_{33} 
  \end{bmatrix}
%  \label{}
\end{align}
where
\begin{align*}
  {A_{11}}_{\ii, \jj} &=   \int_{\Omega} 
          \partial_{x_3} \Psi^{1}_{\jj} \partial_{x_3} \Psi^{1}_{\ii} 
        + \partial_{x_2} \Psi^{1}_{\jj} \partial_{x_2} \Psi^{1}_{\ii}
        \dd \mathbf{x} 
                       +   \int_{\Omega} 
                       \tau \Psi^{1}_{\jj} \Psi^{1}_{\ii} 
        \dd \mathbf{x} 
  \\
  {A_{12}}_{\ii, \jj} &= - \int_{\Omega} 
          \partial_{x_2} \Psi^{1}_{\jj} \partial_{x_1} \Psi^{2}_{\ii}
        \dd \mathbf{x} 
  \\
  {A_{13}}_{\ii, \jj} &= -  \int_{\Omega} 
         \partial_{x_3} \Psi^{1}_{\jj} \partial_{x_1} \Psi^{3}_{\ii}
        \dd \mathbf{x} 
  \\
%  {A_{21}}_{\ii, \jj} &= -   \int_{\Omega} 
%        \partial_{x_1} \Psi^{2}_{\jj} ~ \partial_{x_2} \Psi^{1}_{\jj}
%        \dd \mathbf{x} 
%  \\
  {A_{22}}_{\ii, \jj} &=   \int_{\Omega} 
         \partial_{x_3} \Psi^{2}_{\jj} ~ \partial_{x_3} \Psi^{2}_{\ii} 
       + \partial_{x_1} \Psi^{2}_{\jj} ~ \partial_{x_1} \Psi^{2}_{\ii}
        \dd \mathbf{x} 
                       +   \int_{\Omega} 
                       \tau \Psi^{2}_{\jj} \Psi^{2}_{\ii} 
        \dd \mathbf{x} 
  \\
  {A_{23}}_{\ii, \jj} &=  -  \int_{\Omega} 
       \partial_{x_3} \Psi^{2}_{\jj} ~  \partial_{x_2} \Psi^{3}_{\ii} 
        \dd \mathbf{x} 
  \\
%  {A_{31}}_{\ii, \jj} &= -   \int_{\Omega} 
%        \partial_{x_1} \Psi^{3}_{\jj} ~ \partial_{x_3} \Psi^{1}_{\ii} 
%        \dd \mathbf{x} 
%  \\
%  {A_{32}}_{\ii, \jj} &= -   \int_{\Omega} 
%       \partial_{x_2} \Psi^{3}_{\jj} ~ \partial_{x_3} \Psi^{2}_{\ii} 
%        \dd \mathbf{x} 
%  \\
  {A_{33}}_{\ii, \jj} &=   \int_{\Omega} 
         \partial_{x_2} \Psi^{3}_{\jj} ~ \partial_{x_2} \Psi^{3}_{\ii} 
       + \partial_{x_1} \Psi^{3}_{\jj} ~ \partial_{x_1} \Psi^{3}_{\ii}
        \dd \mathbf{x} 
                       +   \int_{\Omega} 
                       \tau \Psi^{3}_{\jj} \Psi^{3}_{\ii} 
        \dd \mathbf{x} 
\end{align*}
For the right hand side, the entries associated to each component of the vector $\ff$ are given by
\begin{align*}
    F_{1, i} & = \int_{\Omega} \ff_1 \Psi^1_{i} \dd \mathbf{x} \\
    F_{2, i} & = \int_{\Omega} \ff_2 \Psi^2_{i} \dd \mathbf{x} \\
    F_{3, i} & = \int_{\Omega} \ff_3 \Psi^3_{i} \dd \mathbf{x} 
\end{align*}
Hence, we have,
\begin{align*}
  \begin{cases}
    {A_{11}}_{\ii, \jj} &= \left( D_1 \otimes M_2 \otimes K_3 \right)_{\ii \jj} 
                         + \left( D_1 \otimes K_2 \otimes M_3 \right)_{\ii \jj} 
                         + \tau \left( D_1 \otimes M_2 \otimes M_3 \right)_{\ii \jj} 
    \\
    {A_{12}}_{\ii, \jj} &= - \left( R_1 \otimes R_2^T \otimes M_3 \right)_{\ii \jj} 
    \\
    {A_{13}}_{\ii, \jj} &= - \left( R_1 \otimes M_2 \otimes R_3^T \right)_{\ii \jj} 
    \\
    {A_{22}}_{\ii, \jj} & = \left( M_1 \otimes D_2 \otimes K_3 \right)_{\ii \jj} 
                         + \left( K_1 \otimes D_2 \otimes M_3 \right)_{\ii \jj} 
                         + \tau \left( M_1 \otimes D_2 \otimes M_3 \right)_{\ii \jj} 
    \\
    {A_{23}}_{\ii, \jj} &= - \left( M_1 \otimes R_2 \otimes R_3^T \right)_{\ii \jj} 
    \\
    {A_{33}}_{\ii, \jj} &= \left( M_1 \otimes K_2 \otimes D_3 \right)_{\ii \jj} 
                         + \left( K_1 \otimes M_2 \otimes D_3 \right)_{\ii \jj} 
                         + \tau \left( M_1 \otimes M_2 \otimes D_3 \right)_{\ii \jj} 
  \end{cases}
\end{align*}
Therefor, we have the following matrix form 
\begin{align}
  \mathcal{A}^\tau = \mathcal{A}^0 + \tau \mathcal{M}
%  \label{}
\end{align}
where
\begin{align}
  \mathcal{A}^0 = 
  \begin{bmatrix}
    D_1 \otimes M_2 \otimes K_3 + D_1 \otimes K_2 \otimes M_3  &                          - R_1 \otimes R_2^T \otimes M_3   &  - R_1 \otimes M_2 \otimes R_3^T \\
                               - R_1^T \otimes R_2 \otimes M_3 & M_1 \otimes D_2 \otimes K_3 + K_1 \otimes D_2 \otimes M_3  &  - M_1 \otimes R_2 \otimes R_3^T \\
                               - R_1^T \otimes M_2 \otimes R_3 &                          - M_1 \otimes R_2^T \otimes R_3   & M_1 \otimes K_2 \otimes D_3 + K_1 \otimes M_2 \otimes D_3  
  \end{bmatrix}
  \label{eq:matrix-A0-3d}
\end{align}
and
\begin{align}
  \mathcal{M} = 
  \begin{bmatrix}
    D_1 \otimes M_2 \otimes M_3   & 0   &  0 \\
    0 & M_1 \otimes D_2 \otimes M_3   &  0 \\
    0 & 0 &  M_1 \otimes M_2 \otimes D_3 
  \end{bmatrix}
  \label{eq:matrix-M-3d}
\end{align}
