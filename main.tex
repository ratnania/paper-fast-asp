\documentclass[11pt]{article}
\usepackage[margin=2cm]{geometry}
\usepackage{graphicx}
\usepackage{caption}
\usepackage{subcaption}
\usepackage{hyperref}
\usepackage{amsmath}
\usepackage{amsthm}
\usepackage{amssymb}
\everymath{\displaystyle}
\usepackage{bm}
%\usepackage{xcolor}
\newtheorem{theorem}{Theorem}[section]
\newtheorem{definition}{Definition}[section]
\newtheorem{proposition}{Proposition}[section]
\newtheorem{lemma}{Lemma}[section]
\newtheorem{remark}{Remark}[section]
\usepackage[table,xcdraw]{xcolor}
\usepackage{multirow}


% **** ARA
\usepackage{pdflscape}
\usepackage{siunitx}
\newcommand{\cnum}[1]{\textcolor{red}{\num{#1}}} % 
\usepackage[ruled,vlined,linesnumbered]{algorithm2e}
\usepackage[toc,page]{appendix}
\newcommand{\CG}{\texttt{CG}~}
\newcommand{\PCG}{\texttt{PCG}~}
\newcommand{\BICGSTAB}{\texttt{BICGSTAB}~}
\newcommand{\ASP}{\texttt{ASP}~}
\newcommand{\todo}[1]{\textcolor{red}{#1}}
% ****

% ...................................................................
\newcommand{\cmark}{\ding{51}}%
\newcommand{\xmark}{\ding{55}}%
\newcommand{\mmark}{\ding{169}}%
\newcommand{\thetatheta}{\boldsymbol{\theta}}
\newcommand{\glteq}{\sim_{\rm GLT}}
\newcommand{\gltgrad}{\boldsymbol{\delta [\mathfrak{m}_{p}]}}
\newcommand{\gltmm}{\boldsymbol{\mathfrak{m}}_{\mathbf{p}}}
\newcommand{\TrialS}{\mathcal{T}(\mathcal{S}_n^p)}
\newcommand{\TestS}{\mathcal{T}^{\prime}(\mathcal{S}_n^p)}

\newcommand{\fracp}[2]{\frac{\partial #1}{\partial #2}}
\newcommand{\ut}{\boldsymbol{\tau}} % unit tangent
\newcommand{\un}{\boldsymbol{\nu}}  % unit normal
\newcommand{\unh}{\hat{\boldsymbol{\nu}}}  % unit normal on reference element


\newcommand{\dd}{\,{\rm d} }
\newcommand{\UU}{\bm{U}}
\newcommand{\EE}{\bm{E}}
\newcommand{\FF}{\bm{F}}
\newcommand{\GG}{\bm{G}}
\newcommand{\JJ}{\bm{J}}
\newcommand{\HH}{\bm{H}}
\newcommand{\DD}{\bm{D}}
\newcommand{\BB}{\bm{B}}
\newcommand{\V}{\bm{V}}
\newcommand{\ff}{\bm{f}}
\newcommand{\ee}{\bm{e}}
\newcommand{\pp}{\bm{p}}
\newcommand{\uu}{\bm{u}}
\newcommand{\vv}{\bm{v}}
\newcommand{\ww}{\bm{w}}
\newcommand{\xx}{\bm{x}}
\newcommand{\ii}{\bm{i}}
\newcommand{\jj}{\bm{j}}
\newcommand{\kk}{\bm{k}}
\newcommand{\ttheta}{\bm{\theta}}
\newcommand{\etaeta}{\boldsymbol{\eta}}
\newcommand{\mumu}{\boldsymbol{\mu}}
\newcommand{\xixi}{\boldsymbol{\xi}}
\newcommand{\Nip}{{N_{\mumu,}^{\pp}}_{\ii}}
\newcommand{\Njp}{{N_{\mumu,}^{\pp}}_{\jj}}

\newcommand{\Nione}{{N_{\mumu_1,}^{p_1}}_{i_1}}
\newcommand{\Njone}{{N_{\mumu_1,}^{p_1}}_{j_1}}
\newcommand{\Nitwo}{{N_{\mumu_2,}^{p_2}}_{i_2}}
\newcommand{\Njtwo}{{N_{\mumu_2,}^{p_2}}_{j_2}}
\newcommand{\Nitre}{{N_{\mumu_3,}^{p_3}}_{i_3}}
\newcommand{\Njtre}{{N_{\mumu_3,}^{p_3}}_{j_3}}
\newcommand{\Nis}{{N_{\mumu_s,}^{p_s}}_{i_s}}
\newcommand{\Njs}{{N_{\mumu_s,}^{p_s}}_{j_s}}

\newcommand{\Mione}{{M_{\mumu_1-1,}^{p_1-1}}_{i_1}}
\newcommand{\Mjone}{{M_{\mumu_1-1,}^{p_1-1}}_{j_1}}
\newcommand{\Mitwo}{{M_{\mumu_2-1,}^{p_2-1}}_{i_2}}
\newcommand{\Mjtwo}{{M_{\mumu_2-1,}^{p_2-1}}_{j_2}}
\newcommand{\Mitre}{{M_{\mumu_3-1,}^{p_3-1}}_{i_3}}
\newcommand{\Mjtre}{{M_{\mumu_3-1,}^{p_3-1}}_{j_3}}
\newcommand{\Mis}{{M_{\mumu_s-1,}^{p_s-1}}_{i_s}}
\newcommand{\Mjs}{{M_{\mumu_s-1,}^{p_s-1}}_{j_s}}

\newcommand{\eps}{\varepsilon}
\newcommand{\diffcolor}{\textcolor{blue}}
\newcommand{\twofluidcolor}{\textcolor{magenta}}
\newcommand{\termsourcecolor}{\textcolor{darkgreen}}
\newcommand{\Curl}{\nabla \times}
\newcommand{\Adv}{\cdot \nabla}
\newcommand{\Diffp}{\nabla \cdot (D_p \nabla p)}
\newcommand{\Diffu}{\nabla \cdot (D_u \nabla \mathbf{u})}
\newcommand{\veca}{\bm{a}}
\newcommand{\nn}{\bm{n}}
\newcommand{\dotphi}{\dot{\phi}}
\newcommand{\PsiPsi}{\boldsymbol{\Psi}}
\newcommand{\boldbeta}{\boldsymbol{\beta}}
\newcommand{\boldtheta}{\boldsymbol{\theta}}
\newcommand{\intline}{\int_{\mathbb{R}}}
\newcommand{\distreig}{\sim_{\lambda}}
\newcommand{\distrsv}{\sim_{\sigma}}
\newcommand{\Rotv}{\boldsymbol{\nabla} \times}
\newcommand{\Rots}{\nabla \times}
\newcommand{\Div}{\nabla \cdot}
\newcommand{\Grad}{\boldsymbol{\nabla}}
\newcommand{\Ker}[1]{\mbox{Ker}~ #1}
% ...................................................................

% ...................................................................
\newcommand{\Gradh}{\mathbb{G}}
\newcommand{\Curlvh}{\pmb{\mathbb{C}}}
\newcommand{\Curlh}{\mathbb{C}}
\newcommand{\Divh}{\mathbb{D}}

\newcommand{\Hgrad}{H^1(\Omega)}
\newcommand{\Hgradv}{\bm{H}^1(\Omega)}
\newcommand{\Hcurl}{\bm{H}(\mbox{curl}, \Omega)}
\newcommand{\Hdiv}{\bm{H}(\mbox{div}, \Omega)}
\newcommand{\Ltwo}{L^2(\Omega)}
\newcommand{\Ltwov}{\bm{L}^2(\Omega)}
\newcommand{\Hgradzero}{H^1_0(\Omega)}
\newcommand{\Hgradvzero}{\bm{H}^1_0(\Omega)}
\newcommand{\Hcurlzero}{\bm{H}_0(\mbox{curl}, \Omega)}
\newcommand{\Hdivzero}{\bm{H}_0(\mbox{div}, \Omega)}
\newcommand{\Ltwozero}{L^2_0(\Omega)}
\newcommand{\Ltwovzero}{\bm{L}^2_0(\Omega)}
\newcommand{\Vgrad}{V_h(\mbox{grad}, \Omega)}
\newcommand{\Vgradv}{\bm{V}_h(\mbox{grad}, \Omega)}
\newcommand{\Vcurl}{\bm{V}_h(\mbox{curl}, \Omega)}
\newcommand{\Vdiv}{\bm{V}_h(\mbox{div}, \Omega)}
\newcommand{\Vltwo}{V_h(L^2, \Omega)}
\newcommand{\Cinfinity}{\mathcal{C}^{\infty}(\Omega)}

%\newcommand{\Pigrad}{\boldsymbol{\Pi}_h^{\mbox{\footnotesize{grad}}}}
%\newcommand{\Picurl}{\boldsymbol{\Pi}_h^{\mbox{\footnotesize{curl}}}}
%\newcommand{\Pidiv}{ \boldsymbol{\Pi}_h^{\mbox{\footnotesize{div}}}}
%\newcommand{\Piltwo}{\boldsymbol{\Pi}_h^{\footnotesize{L^2}}}

\newcommand{\Picurlx}{\boldsymbol{\Pi}_h^{\mbox{\footnotesize{curl},1}}}
\newcommand{\Picurly}{\boldsymbol{\Pi}_h^{\mbox{\footnotesize{curl},2}}}
\newcommand{\Picurlz}{\boldsymbol{\Pi}_h^{\mbox{\footnotesize{curl},3}}}
\newcommand{\Picurlk}{\boldsymbol{\Pi}_h^{\mbox{\footnotesize{curl},k}}}

%\newcommand{\Pigrad}{\pi_h^{\footnotesize{0}}}
%\newcommand{\Picurl}{\pi_h^{\footnotesize{1}}}
%\newcommand{\Pidiv}{\pi_h^{\footnotesize{2}}}
%\newcommand{\Piltwo}{\pi_h^{\footnotesize{3}}}

\newcommand{\Pigrad}{P_h^{\mbox{\footnotesize{grad}}}}
\newcommand{\Picurl}{P_h^{\mbox{\footnotesize{curl}}}}
\newcommand{\Pidiv}{P_h^{\mbox{\footnotesize{div}}}}
\newcommand{\Piltwo}{P_h^{\footnotesize{L^2}}}

\newcommand{\diff}{\mathrm{d}\,}


\newcommand{\HgradLogical}{H^1(\hat{\Omega})}
\newcommand{\HcurlLogical}{{H}(\mbox{curl}, \hat{\Omega})}
\newcommand{\HdivLogical}{{H}(\mbox{div}, \hat{\Omega})}
\newcommand{\LtwoLogical}{L^2(\hat{\Omega})}
\newcommand{\igrad}{\imath^{0}}
\newcommand{\icurl}{\imath^{1}}
\newcommand{\idiv}{\imath^{2}}
\newcommand{\iltwo}{\imath^{3}}
\newcommand{\VgradLogical}{V_h(\mbox{grad}, \hat{\Omega})}
\newcommand{\VcurlLogical}{{V}_h(\mbox{curl}, \hat{\Omega})}
\newcommand{\VdivLogical}{{V}_h(\mbox{div}, \hat{\Omega})}
\newcommand{\VltwoLogical}{V_h(L^2, \hat{\Omega})}

\newcommand{\Vgradspline}{\mathcal{S}^{p,p,p}}
\newcommand{\Vltwospline}{\mathcal{S}^{p-1,p-1,p-1}}
\newcommand{\Vcurlspline}{
  \begin{pmatrix}
    \mathcal{S}^{p-1,p,p} \\ 
    \mathcal{S}^{p,p-1,p} \\ 
    \mathcal{S}^{p,p,p-1}
  \end{pmatrix}
}
\newcommand{\Vdivspline}{
  \begin{pmatrix}
    \mathcal{S}^{p,p-1,p-1} \\ 
    \mathcal{S}^{p-1,p,p-1} \\ 
    \mathcal{S}^{p-1,p-1,p}
  \end{pmatrix}
}

\newcommand{\matrixIdentity}{\mathbb{I}}

\title{Efficient and Fast Auxiliary Splines Space Preconditioning B-Splines Mixed Finite Elements}

\author{A. El-Akri$^{1}$, Y. Gu\c{c}lu$^{2}$, N. Hamid$^{1}$, I. Kissami$^{1}$, N. Ouhaddou$^{1}$, A. Ratnani$^{1}$ 
\\
\small{\textit{$^1$MSDA, Mohammed VI Polytechnic University, Green City, Morocco}}
\\
\small{\textit{$^2$Max-Planck-Institut fur Plasmaphysik, Garching, Germany}}
}

\begin{document}

\maketitle

\begin{abstract}
In this paper, we derive an optimal and robust preconditioning strategy for $\bm{H}(\textbf{curl},\Omega)-$elliptic problems, on unit square/cube domains, in the case of regular B-Splines Finite Elements. Our method is based on Auxiliary Space Preconditioning method, developed by Hiptmair et al. 

\end{abstract}

\textcolor{red}{TODO: 
  \begin{itemize}
    \item   
    \item   
    \item   
    \item   
  \end{itemize}
}

% ********************************************
% ARA: To be removed later
\newpage
\tableofcontents
\newpage
% ********************************************

% ********************************************
\section{Introduction}
\todo{TODO: rewrite the introduction}

%The {\em IsoGeometric Analysis}, or IGA for short, is a mathematical approach that incorporates two disciplines, namely {\em Finite Element Methods} (FEMs)  and {\em Computer-Aided Design} (CAD) in order to design and analyze the numerical approximation of {\em Partial Differential Equations} (PDEs). Similarly to the FEM, solving a problem in partial differential equations by the isogeometric methods consist of a variational formulation of the problem and the specification of a finite-dimensional subspace of the function space where the solution belongs to, and then the construction of basis functions for this subspace.  However, in contrast to FEM, these functions are the same functions used for the presentation of the underlying domain which usually employs B-spline or more generally complex Non-Uniform Rational B-splines (NURBS) functions (commonly used in CAD world). This strategy leads at least to two main advantages, with respect to FEM: first isogeometric analysis employs the exact geometry and there is no geometric approximation error, and secondly, the use of $B$-spline functions, instead of the standard Lagrange and Hermite polynomials based finite element method, has the advantage of being more suitable for higher $C^p$-continuous interpolation.
%
%The literature on IGA has rapidly developed in the last years
%starting from the work Ref., and by now there is large contributions on the field. For instance, this new approach has been used in electromagnetism Refs., in incompressible fluid dynamics Ref., in fluid-structure interaction Refs., in structural and contact mechanics Refs, in Plasmas physics problems Ref. and in the Kinetic systems Refs. and many others. For an extensive overview we reefer the reader to the review paper Ref., see also Refs.
%
%Despite the large success of the method, it is worth mentioning, however, that one critical aspect is that when we are dealing with higher-order IGA finite elements, the use of B-spline functions generates matrices  which are denser than the FEM matrices. In general, this leads to condition number that is not uniformly bounded with respect to discretization parameter $h$ and can even grow rapidly  when $h$ goes to zero,  this fact has been observed numerically by various authors and proved analytically in Ref. As a consequence, the employing of standard numerical solution methods for IGA discrete systems may fail and preconditioning is necessary to obtain convergence in a reasonable amount of time.
%
%To overcome the difficulties given by the observed ill-onditioning, several techniques have been developed in the literature. For instance, {\em additive Schwarz algorithms} were successfully  implemented in Ref. and a good convergence behavior of the preconditioners with respect to discretization parameters  $h$, $p$ and $k$ was observed. The analysis in
%Ref. done in the context of scalar elliptic problems while a technical generalization to the system of linear elasticity has developed later in Ref. The hints proposed in Refs. have been generalized in several directions 
%
%\newpage

We consider the following variational problems, on a bounded domain $\Omega \subset \mathbb{R}^d$, with $d \in \{2, 3\}$:   
\begin{subequations}
  \begin{equation}\label{eq:curl-variational-problem}
      \bm{u} \in \Hcurlzero: \quad
      \left( \Curl \uu, \Curl \vv \right)_{\mathbb{L}^2(\Omega)} + \mu \left( \uu,\vv \right)_{\mathbb{L}^2(\Omega)}=\left( \ff, \vv \right)_{\mathbb{L}^2(\Omega)} \quad \forall \bm{v} \in \Hcurlzero 
  \end{equation}
  \begin{equation}\label{eq:div-variational-problem}
      \bm{u} \in \Hdivzero: \quad
      \left( \Div \uu, \Div \vv \right)_{\mathbb{L}^2(\Omega)} + \mu \left( \uu,\vv \right)_{\mathbb{L}^2(\Omega)}=\left( \ff, \vv \right)_{\mathbb{L}^2(\Omega)} \quad \forall \bm{v} \in \Hdivzero 
  \end{equation}
%  \label{}
\end{subequations}
where $0<\mu \ll 1$ and $\bm{f} \in \left(L^2(\Omega)\right)^3$. 
%We recall that 
%\begin{equation*}
%\bm{H}(\textbf{curl},\Omega) = \left \{ \bm{u} \in  \left(L^2(\Omega)\right)^3 \,:\, \textbf{curl}\ (\bm{u}) \in \left(L^2(\Omega)\right)^3\right\},
%\end{equation*}
%equipped with the norm
%\begin{equation*}
%\|\bm{u}\|_{\bm{H}(\textbf{curl},\Omega)}^2=\|\bm{u}\|_{\left(L^2(\Omega)\right)^3}^2+\|\textbf{curl}\, (\bm{u})\|_{\left(L^2(\Omega)\right)^3}^2.
%\end{equation*}

\section{Preliminaries}
In this section we fix some notations and we recall some preliminary results which will be used in the paper. In details, in Subsection \ref{subsec:functional-spaces} after introducing the functional spaces we recall a theoretical result corresponds to so-called regular decomposition of a three dimensional vector filed, which will be fundamental for our study of the discrete curl-curl problem discussed in Section \ref{sec:asp}. Then we summarize in Subsection \ref{subsec:Auxiliary-Space-Preconditioning-Method} the essentials of the abstract theory of Auxiliary Space Preconditioning (ASP) method. The last subsection is devoted to IsoGeometric Analysis (IGA) with some relevant properties of IsoGeometric spaces.


    
\subsection{Functional Spaces: Notation and Results}\label{subsec:functional-spaces}
Throughout this paper we shell work with {\em Sobolev spaces} and recall here some standard notations. For a more detailed presentation we refer to \cite{adams2003sobolev,girault2012finite,monk2003finite}. For a bounded domain $\Omega \subset \mathbb{R}^3$, the Hilbert space $L^2(\Omega)$ denotes the {\em Lebesgue square-integrable functions} on $\Omega$ equipped with standard $L^2(\Omega)$ norm. Given a positive integer $s$, we denote by $H^s(\Omega)$ the Sobolev space of order $s$ on $\Omega$, namely the space of functions in $L^2(\Omega)$ with $s$th-order derivatives, in the sense of distributions, also in $L^2(\Omega)$ endowed with the standard norm $\|\cdot\|_{H^s(\Omega)}$. Note that, by definition we have $H^0(\Omega)=L^2(\Omega)$. For the correspond vectorial spaces we use the notations $\mathbb{L}^2(\Omega):=\left(L^2(\Omega)\right)^3$ and $\mathbb{H}^s(\Omega):=\left(H^s(\Omega)\right)^3$. However, we shall need also the following Hilbert spaces
\begin{equation*}
\bm{H}(\textbf{curl},\Omega)=\left\{ \bm{u} \in \mathbb{L}^2(\Omega),\: \, \textbf{curl}\,(\bm{u})\in \mathbb{L}^2(\Omega)\right\}, \; \bm{H}(\text{div},\Omega)=\left\{ \bm{v} \in \mathbb{L}^2(\Omega),\: \, \text{div}\,(\bm{v})\in L^2(\Omega)\right\},
\end{equation*}
equipped with the norms
\begin{equation*}
\|\bm{u}\|_{\bm{H}(\textbf{curl},\Omega)}^2=\|\bm{u}\|_{\mathbb{L}^2(\Omega)}^2 + \|\textbf{curl}\,(\bm{u})\|_{\mathbb{L}^2(\Omega)}^2, \; \|\bm{v}\|_{\bm{H}(div,\Omega)}^2=\|\bm{v}\|_{\mathbb{L}^2(\Omega)}^2 + \|\text{div}\,(\bm{v})\|_{L^2(\Omega)}^2,
\end{equation*}
respectively.

We are now ready to provide regular decomposition of space $\bm{H}(\textbf{curl},\Omega)$. However, to keep the presentation focused we assume that {\em $\Omega$ is homotopy equivalent to a ball}. A consequence of this assumption is that the following DeRham complex 
\begin{align*}
    \begin{array}{ccccc}
   H^1(\Omega) & \xrightarrow{\quad \textbf{grad} \quad } & \bm{H}(\textbf{curl},\Omega) & \xrightarrow{\quad \textbf{curl} \quad } & \bm{H}(\text{div},\Omega)
  \end{array}
\end{align*}
is exact, meaning that $\textbf{grad} \left( H^1(\Omega)\right)=\text{ker}(\textbf{curl})$


The following theorem is essentially, for instance see \cite{pasciak2002overlapping,zhao2004analysis}.

\begin{theorem}[Regular decomposition of $\mathbf{H}(\textbf{curl},\Omega)$]\label{regular-decomposition-of-H(curl)}
For each $\mathbf{u} \in \bm{H}(\textbf{curl},\Omega)$, there exist $\bm{\varphi} \in \mathbb{H}^1(\Omega)$ and $\phi \in H^1(\Omega)$ such that
\begin{itemize}
\item[(i)] $\bm{u}=\bm{\varphi}+ \textbf{grad}\, (\phi)$;

%$\textbf{curl}\,(\bm{u})=\textbf{curl}\,(\bm{\varphi})$;

\item[(ii)] $\|\bm{\varphi}\|_{\mathbb{H}^1(\Omega)} \leq C\|\textbf{curl}\,(\bm{u})\|_{\mathbb{L}^2(\Omega)}$;

\item[(iii)] $\|\bm{\varphi}\|_{\mathbb{L}^2(\Omega)} \leq C \|\bm{u}\|_{\mathbb{L}^2(\Omega)}$,
\end{itemize}
where $C$ is a positive constant depending only on the shape of $\Omega$.
\end{theorem}

\begin{remark}
From decomposition $(i)$, since $\textbf{grad}\, (\phi) \in \text{ker}\, (\textbf{curl})$, we deduce that $\textbf{curl}\,(\bm{u})=\textbf{curl}\,(\bm{\varphi})$.
\end{remark}

\begin{remark}
In the general case in which $\Omega$ is simply a three-dimensional polyhedral domain, the result of the theorem can be generalized  as follows: For each $\bm{u} \in \bm{H}(\textbf{curl},\Omega)$, there exists $\bm{\varphi} \in \mathbb{H}^1(\Omega)$ such that $\textbf{curl}\,(\bm{u})=\textbf{curl}\,(\bm{\varphi})$ with estimates $(ii)-(iii)$. For the proof see \cite{kolev2009parallel}.
\end{remark}

In Section \ref{sec:asp}, in Theorem \ref{thm:stable-Hitpmair-Xu-decomposition}, we will show a discrete version of Theorem \ref{regular-decomposition-of-H(curl)}.
 
\subsection{Auxiliary Space Preconditioning (ASP) Method}\label{subsec:Auxiliary-Space-Preconditioning-Method}
For completeness, we give in this subsection the main ingredients of the {\em ASP method}, for a general discussion of the method the reader is referred to \cite{chen2015auxiliary,hiptmair2006auxiliary,hiptmair2007nodal,kolev2008auxiliary,kolev2009parallel,nepomnyaschikh1991decomposition,xu1992iterative} and references therein. 

Let $V$ be a Hilbert space and $a\,:\, V \times V \longrightarrow \mathbb{R}$ some inner product defined on $V$. The main ingredients of the method are {\em auxiliary spaces}, {\em transfer operators} and what is called the {\em smoother}. The auxiliary spaces are Hilbert spaces, say $W_1$ and $W_2$, equipped  with some inner products ${a}_1 \,:\, W_1 \times W_1 \longrightarrow \mathbb{R}$ and ${a}_2 \,:\, W_2 \times W_2 \longrightarrow  \mathbb{R}$. The transfer operators are linear operators $\pi_1\,:\, W_1 \longrightarrow V$ and $\pi_2 \,:\, W_1 \longrightarrow V$ transferring the auxiliary spaces to the original space $V$. The smoother is simply an inner product $r\,:\, V \times V \longrightarrow \mathbb{R}$ defined on $V$. With these considerations the ASP preconditioner is given by
\begin{equation*}\label{eq:ASP-preconditione}
B=R^{-1} + \pi_1 \circ {A}_1^{-1} \circ \pi_1^*+\pi_2 \circ {A}_2^{-1} \circ \pi_2^*,
\end{equation*} 
where $^*$ stands for the adjoint operator and $R$, $A_1$, $A_2$ are the linear operators related to the inner products $r$, $a_1$ and $a_2$ respectively. 

Under suitable assumptions, we prove that $B$ is indeed a  preconditioner for $A$, more precisely we have the following result
\begin{theorem}{\cite[Theorem 2.2]{hiptmair2007nodal}} \label{th:ASP-lemma}
Assume that there are some nonnegative constants $\beta_1$, $\beta_2$, $\gamma$ and $\eta$ such that
\begin{itemize}
\item[(i)] The continuity of $\pi_1$ and $\pi_2$ with respect to the graph norms:
\begin{equation*}
a\left(\pi_1 (w_1),\pi_1 (w_1)\right)^{1/2} \leq \beta_1 \left( a_1(w_1,w_1)^{1/2}+ a_2(w_1,w_1)^{1/2}\right),
\end{equation*} 
\begin{equation*}
a\left(\pi_2 (w_2),\pi_2 (w_2)\right)^{1/2} \leq \beta_2 \left( a_1(w_2,w_2)^{1/2}+ a_2(w_2,w_2)^{1/2}\right), 
\end{equation*} 
for all $w_1 \in W_1$ and $w_2 \in W_2$;

\item[(ii)]  The continuity of $r^{-1}$:
\begin{equation*}
a(v,v)^{1/2} \leq \gamma \,r(v,v)^{1/2},
\end{equation*}
for all $v \in V$;

\item[(iii)] Existence of a stable decomposition of $V$: for each $v \in V$, there exist $\widetilde{v} \in V$, $w_1 \in W_1$ and $w_2 \in W_2$ such that 
\begin{equation*}
v=\widetilde{v}+\pi_1 w_1 + \pi_2 w_2,
\end{equation*}
with estimate 
\begin{equation*}
r(\widetilde{v},\widetilde{v}) + a_1(w_1,w_1) + a_2(w_2,w_2) \leq \eta \, a(v,v).
\end{equation*} 
\end{itemize}
Then we have the following estimate for the {\em spectral condition number} of the preconditioned operator
\begin{equation*}
\kappa(BA) \leq \eta^2 (\beta_1^2+\beta_2^2).
\end{equation*}
\end{theorem}

The above result make in evidence the central importance of stable regular decompositions on the construction of an efficient auxiliary space preconditioning. In this work, we focus on the discrete case and hence the regular decomposition of Theorem \ref{regular-decomposition-of-H(curl)} has to be adapted at the discrete level. As a first step we introduce in the next section the discrete spaces.

\subsection{IsoGeometric Spaces}\label{subsec-IGA}
In this section we introduce a discrete counterparts of functional spaces $L^2(\Omega)$, $\bm{H}(\textbf{curl},\Omega)$, $\bm{H}(\text{div},\Omega)$, this is done in the context of IsoGeometric Analysis (IGA) \cite{bazilevs2006isogeometric,buffa2011isogeometric,cottrell2009isogeometric,da2014mathematical,hughes2005isogeometric}. 

Accordingly, we start by recalling some basic properties of $B$-spline functions. For a basic introduction to the subject the reader is referred to standard textbooks \cite{cohen2001geometric,farin1999nurbs,farin2002curves,gu2008computational,piegl1996nurbs,prautzsch2002bezier,rogers2001introduction,schumaker2007spline}. Given a knot vector $T=(t_1,t_2,\ldots,t_{m})$, namely a nondecreasing sequence of real numbers, the $i$-th $B$-spline of order $p \in \mathbb{N}$ is defined recursively  by the following {\em Cox–de Boor formula} 
\begin{equation*}
B_{i,0}(t) = 
\begin{cases}
1 \quad &\text{if } t_i \leq t < t_{i+1},\\
0 \quad &\text{otherwise}
\end{cases}
\end{equation*}
\begin{equation*}
B_{i,p}(t)=\frac{t - t_i}{t_{i+p}-t_i} B_{i,p-1}(t) + \frac{t_{i+p+1} - t}{t_{i+p+1}-t_{i+1}} B_{i+1,p-1}(t), 
\end{equation*}
for $i=1, \ldots, n$ with $n=m-p-1$. Following \cite{buffa2011isogeometric}, we introduce also the vector $U=(u_1,\ldots,u_N)$ of breakpoints where $N$ is the number of knots without repetition and the regularity vector $\bm{\alpha}=(\alpha_1, \ldots, \alpha_N) \in \mathbb{N}^N$ in such a way that for each $i \in \{ 1,\ldots,N\}$, the $B$-spline function $B_{i,p}$ is continuously derivable at the breakpoint $u_i$. Note that $\alpha_i=p-r_i$ where $r_i$ is the multiplicity of the break point $u_i$. However, we will only consider  {\em non-periodic} knot vectors
\begin{equation*}
T=(\underbrace{0,\ldots,0}_{p+1}, t_{p+2}, \ldots, t_{m-p-1}, \underbrace{1,\ldots,1}_{p+1}),
\end{equation*}  
and we suppose that $0 \leq r_i \leq p+1$. In this way we guarantee that $-1 \leq \alpha_i \leq p+1$ where the minimal regularity $\alpha_i=-1$ corresponds to a discontinuity at knot $u_i$. We also introduce the {\em Schoenberg space} 
\begin{equation*}
\mathcal{S}^p_{\bm{\alpha}}= span\left\{ B_{i,p} \,:\, i=1,\ldots,n\right\}.
\end{equation*} 
This definition is generalized to the multivariate case $\Omega=(0,1)^3$ by {\em tensorization}: With a tridirectional knot vector $\bm{T}=T_1 \times T_2 \times T_3$ at hand, where
\begin{equation*}
T_i=(\underbrace{0,\ldots,0}_{p_i+1}, t_{i,p_i+2}, \ldots, t_{i,m_i-p_i-1}, \underbrace{1,\ldots,1}_{p_i+1}),\quad m_i, p_i \in \mathbb{N}, \; i=1,2,3,
\end{equation*} 
is an open univariate knot vector, the {\em three dimensional  Schoenberg space} is defined by 
\begin{equation*}
\bm{\mathcal{S}}^{p_1,p_2,p_3}_{\bm{\alpha}_1,\bm{\alpha}_2,\bm{\alpha}_3}= \mathcal{S}^{p_1}_{\bm{\alpha}_1} \otimes \mathcal{S}^{p_2}_{\bm{\alpha}_2} \otimes \mathcal{S}^{p_3}_{\bm{\alpha}_3},
\end{equation*} 
where $\bm{\alpha}_i$ is the regularity vector related to knot $T_i$, with $i=1,2,3$. However, we shall also assume our mesh to be {\em locally quasi-uniform}, meaning, there exists a constant $\theta \geq 1$ such that for all $i\in \{1,2,3\}$ we have 
\begin{equation*}
\frac{1}{\theta} \leq \frac{h_{i,j_i}}{h_{i,j_i+1}} \leq \theta, \quad j_i=1, \ldots, N_i-2,
\end{equation*}
where $N_i$ is the number of $T_i$-knots without repetition and $h_{i,j_i}=t_{i,j_i+1}-t_{i,k_{j_i}}$, with $k_{j_i}=\max \{l\,:\,t_l < t_{i,j_i+1}\}$.

With these notations the IsoGeometric spaces read \cite{buffa2011isogeometric,da2014mathematical}
\begin{equation}\label{eq:IsoGeometric-spaces}
\begin{cases}
V_h(\textbf{grad},\Omega):=\bm{\mathcal{S}}^{p_1,p_2,p_3}_{\bm{\alpha}_1,\bm{\alpha_2},\bm{\alpha_3}}\\
\bm{V}_h(\textbf{curl},\Omega):=\bm{\mathcal{S}}^{p_1-1,p_2,p_3}_{\bm{\alpha}_1-1,\bm{\alpha_2},\bm{\alpha_3}} \times \bm{\mathcal{S}}^{p_1,p_2-1,p_3}_{\bm{\alpha}_1,\bm{\alpha_2}-1,\bm{\alpha_3}} \times \bm{\mathcal{S}}^{p_1,p_2,p_3-1}_{\bm{\alpha}_1,\bm{\alpha_2},\bm{\alpha_3}-1}\\
\bm{V}_h(\text{div},\Omega):=\bm{\mathcal{S}}^{p_1,p_2-1,p_3-1}_{\bm{\alpha}_1,\bm{\alpha_2}-1,\bm{\alpha_3}-1} \times \bm{\mathcal{S}}^{p_1-1,p_2,p_3-1}_{\bm{\alpha}_1-1,\bm{\alpha_2},\bm{\alpha_3}-1} \times \bm{\mathcal{S}}^{p_1-1,p_2-1,p_3}_{\bm{\alpha}_1-1,\bm{\alpha_2}-1,\bm{\alpha_3}},
\end{cases}
\end{equation}
where $h$ refers to the global mesh size, i.e $h=\max_{\substack{1 \leq j_i \leq N_i-1 \\ i=1,2,3}}h_{i,j_i}$. These discrete spaces enjoy the following property 
\begin{theorem}{\cite[Theorem 5.3]{buffa2011isogeometric}}
Let $l$ and $s$ be two integers such that $0\leq l \leq s \leq \underline{p}$ and $l \leq \underline{\alpha}$ where $\underline{p}=\min\{p_1,p_2,p_3\}$ and $\underline{\alpha}=\min_{i=1,2,3} \min \bm{\alpha}_i$. Then the following estimates
\begin{equation*}
\|\varphi - \Pi_h^{\textbf{grad}}\varphi\|_{H^l(\Omega)} \leq C h^{s-l}\|\varphi\|_{H^s(\Omega)}, \quad \varphi \in V_h(\textbf{grad},\Omega) \cap H^s(\Omega),
\end{equation*}
\begin{equation*}
\|\bm{u} - \Pi_h^{\textbf{curl}}\bm{u}\|_{\mathbb{H}^l(\Omega)} \leq C h^{s-l}\|\bm{u}\|_{\mathbb{H}^s(\Omega)}, \quad \bm{u} \in \bm{V}_h(\textbf{curl},\Omega) \cap \mathbb{H}^s(\Omega),
\end{equation*}
hold true, where $C$ is a positive constant which is independent   of $h$.
\end{theorem}

Now {\em DeRham diagrams} can be constructed. Among the important properties, one can build specific projectors, what is called {\em quasi interpolation operators}, that make these diagrams commute. We shall start with the univariate case, then extend it by tensor product. For this purpose, we take any locally stable projector $\mathcal{P}_h \,:\, H^1(0,1) \longrightarrow \mathcal{S}^p_{\bm{\alpha}}$, for instance see \cite{schumaker2007spline} for theoretical studies, then we define the corresponding 
histopolation operator by 
\begin{equation*}
\mathcal{Q}_h \phi = \frac{d}{d x} \mathcal{P}_h \left( \int_0^x \phi(t) dt \right), \quad \phi \in L^2(0,1). 
\end{equation*} 
Following the notations above, the quasi interpolation operators are given by  
\begin{equation*}
\Pi^{\textbf{grad}}_h=\mathcal{P}_h \otimes \mathcal{P}_h \otimes \mathcal{P}_h,
\end{equation*}
\begin{equation*}
\Pi^{\textbf{curl}}_h=\left(\mathcal{Q}_h \otimes \mathcal{P}_h \otimes \mathcal{P}_h,\mathcal{P}_h \otimes \mathcal{Q}_h \otimes \mathcal{P}_h,\mathcal{P}_h \otimes \mathcal{P}_h \otimes \mathcal{Q}_h\right),
\end{equation*}
and
\begin{equation*}
\Pi^{div}_h=\left(\mathcal{P}_h \otimes \mathcal{Q}_h \otimes \mathcal{Q}_h,\mathcal{Q}_h \otimes \mathcal{P}_h \otimes \mathcal{Q}_h,\mathcal{Q}_h \otimes \mathcal{Q}_h \otimes \mathcal{P}_h\right).
\end{equation*}

Finally, we shall need the following result
\begin{proposition}{\cite[Proposition 4.5]{buffa2011isogeometric}}
The following diagram
\begin{align}\label{pr-eq:DeRham-diagram}
    \begin{array}{ccccc}
   H^1(\Omega) & \xrightarrow{\quad \textbf{grad} \quad } & \bm{H}(\textbf{curl},\Omega) & \xrightarrow{\quad \textbf{curl} \quad } & \bm{H}(\text{div},\Omega)\\
  \Pi_h^{\textbf{grad}} \Bigg\downarrow &  & \Pi_h^{\textbf{curl}} \Bigg\downarrow &  & \Pi_h^{\text{div}} \Bigg\downarrow \\
  V_h(\textbf{grad},\Omega) & \xrightarrow{\quad \textbf{grad} \quad } & \bm{V}_h(\textbf{curl},\Omega) & \xrightarrow{\quad \textbf{curl} \quad } & \bm{V}_h(\text{div},\Omega)
  \end{array}
\end{align}
commutes and is exact.
\end{proposition}





%\section{Global approximation}
%\subsection{Splines Interpolation}
%Given the set of interpolations points $X:=\{ x_0, \cdots,x_n\}$ and data $Y:=\{ y_0, \cdots,y_n\}$, we aim to find a spline $s$ such that $y_i = s(x_i)$ . The Spline interpolation problems is
%\begin{definition}[Spline interpolation]
%  Find a spline $s := \sum\limits_{j=0}^n c_j N_j^p \in \mathcal{S}_p(T)$ such that
%  \begin{align}
%    s(x_i) = y_{i} \quad \quad 0 \leq i \leq n
%%    \label{}
%  \end{align}
%\end{definition}
%In a matrix form, the Spline interpolation problem writes
%\begin{align}
%  M c = y
%  \label{eq:interpolation-linsys}
%\end{align}
%where $c$ is the unkown vector of the spline coefficients, 
%\begin{align*}
%  c =  
%  \begin{pmatrix}
%    c_0\\
%    c_1\\
%    \vdots    \\ 
%    c_n\\
%  \end{pmatrix}
%\end{align*}
%$M$ is the \textbf{collocation matrix} given by
%\begin{align}
%  M = 
%  \begin{pmatrix}
%    N_0^p(x_0) & \ldots & N_n^p(x_0) \\
%    N_0^p(x_1) & \ldots & N_n^p(x_1) \\
%    \vdots     & \ldots & \vdots \\
%    N_0^p(x_n) & \ldots & N_n^p(x_n) \\
%  \end{pmatrix}
%  \label{eq:collocation-matrix}
%\end{align}
%while $y$ is the given data
%\begin{align*}
%  y =  
%  \begin{pmatrix}
%    y_0\\
%    y_1\\
%    \vdots    \\ 
%    y_n\\
%  \end{pmatrix}
%\end{align*}
%Notice that when interpolating a function $f$, the given data will be $y_i := f(x_i)$.
%\\
%\noindent
%In general the linear system given by the Eq. \ref{eq:interpolation-linsys} is not always solvable. The following result by Whitney-Schoenberg gives a necessary and sufficient condition to ensure that the interpolation problem has a unique solution. 
%\begin{theorem}
%  The collocation matrix is nonsingular if and only if the diagonal elements are positive, \textit{i.e.}
%  \begin{align}
%    N_i^p(x_i) > 0 \quad \forall i \in \{ 0, \ldots, n \}
%    \label{eq:whitney-schoenberg-condition}
%  \end{align}
%  This condition is also equivalent to 
%  \begin{align}
%    t_i < x_i < t_{i+p+1} \quad \forall i \in \{ 0, \ldots, n \}
%    \label{eq:whitney-schoenberg-condition-knots}
%  \end{align}
%  \label{thm:whitney-schoenberg}
%\end{theorem}
\subsection{Splines Histopolation}
In this subsection, we present the histopolation problem and its matrix form.
In opposition to the interpolation problem, where we preserve the values of a function on a given nodes, another interesting way to approximate a function,  is to preserve the integrals between given points, rather than the value of the function on these points. Given the set of interpolations points $X:=\{ x_0, \cdots,x_{n} \}$ and a continuous function $f$, the histopolation problem writes 
\begin{definition}[Spline histopolation]
  Find a spline $s := \sum\limits_{j=1}^{n} c_j N_j^p \in \mathcal{S}_p(T)$ such that
  \begin{align}
    \int_{x_i}^{x_{i+1}} s ~dx = \int_{x_i}^{x_{i+1}} f ~dx \quad \quad 1 \leq i \leq n
%    \label{}
  \end{align}
\end{definition}
\noindent
In a matrix form, the Spline histopolation problem writes $ H c = y $ where $H$ is the \textbf{histopolation matrix}, $c$ is the unkown vector of the spline coefficients and $y$ is the given data, are given by
\begin{align*}
  c =  
  \begin{bmatrix}
    c_1\\
    c_2\\
    \vdots    \\ 
    c_n\\
  \end{bmatrix}
  \quad
  H = 
  \begin{bmatrix}
    \int_{x_0}^{x_{1}}N_1^p ~dx   & \ldots & \int_{x_0}^{x_{1}}N_n^p ~dx   \\
    \int_{x_1}^{x_{2}}N_1^p ~dx   & \ldots & \int_{x_1}^{x_{2}}N_n^p ~dx   \\
    \vdots                        & \ldots &                        \vdots \\
    \int_{x_n}^{x_{n+1}}N_1^p ~dx & \ldots & \int_{x_n}^{x_{n+1}}N_n^p ~dx \\
  \end{bmatrix}
  \quad
  y =  
  \begin{bmatrix}
    \int_{x_0}^{x_{1}}f ~dx\\
    \int_{x_1}^{x_{2}}f ~dx\\
    \vdots    \\ 
    \int_{x_n}^{x_{n+1}}f ~dx\\
  \end{bmatrix}
%  \label{eq:histopolation-matrix}
\end{align*}
We recall that the histopolation matrix $H$ is non singular iff  $t_i < x_i < t_{i+p+1} \quad \forall i \in \{ 0, \ldots, n \}$ \todo{add citation}.
%\begin{theorem}
%  The histopolation matrix is nonsingular if and only if 
%  \begin{align}
%    t_i < x_i < t_{i+p+1} \quad \forall i \in \{ 0, \ldots, n \}
%    \label{eq:whitney-schoenberg-condition-histo-knots}
%  \end{align}
%  \label{thm:whitney-schoenberg-histo}
%\end{theorem}
%\begin{proof}
%  TODO
%\end{proof}

\paragraph{Histopolation using M-Splines} \mbox{}\\
Rather than using the B-Splines for the histopolation problem, one can use the M-Splines. In this case, the histopolation matrix can be computed easily using the following result.
\begin{proposition}
  For every $0 \le i \le  n$ and $0 \le j \le n$, we have
  \begin{align}
    \int_{x_i}^{x_{i+1}} M_j^p(t) ~dt = \sum_{k=0}^{j-1} \left( N_k^p(x_i) - N_k^p(x_{i+1}) \right)  
%    \label{}
  \end{align}
%  \label{}
\end{proposition}
\begin{proof}
Integrating the relation $\frac{d}{dt}N_k^p(t)=M_k^{p}(t)-M_{k+1}^{p}(t)$ on the interval $[x_i, x_{i+1}]$, we have
\begin{align*}
  N_k^p(x_{i+1}) - N_k^p(x_i) = \int_{x_i}^{x_{i+1}} \left( M_k^{p}(t)-M_{k+1}^{p}(t) \right) ~dt 
%  \label{}
\end{align*}
summing the last equation for $k=0$ to $k=j-1$, we get
\begin{align*}
  \sum_{k=0}^{j-1} \left( N_k^p(x_{i+1}) - N_k^p(x_i) \right) 
  &= \int_{x_i}^{x_{i+1}} \sum_{k=0}^{j-1} \left( M_k^{p}(t)-M_{k+1}^{p}(t) \right) ~dt 
  \\
  &= \int_{x_i}^{x_{i+1}} \left( M_0^{p}(t)-M_{j}^{p}(t) \right) ~dt 
%  \label{}
\end{align*}
hence,
\begin{align*}
  \int_{x_i}^{x_{i+1}} M_{j}^{p}(t) ~dt = \sum_{k=0}^{j-1} \left( N_k^p(x_{i}) - N_k^p(x_{i+1}) \right) 
%  \label{}
\end{align*}
\end{proof}
%\begin{remark}
%  The last result gives an optimized implementation for the assembly of the histopolation matrix, since the right hand side term can be computed by accumulating the summation for each $j$.
%\end{remark}

%\paragraph{Commuting diagram} \mbox{}\\
%Let us define the following discrete spline spaces 
%\begin{align}
%  V_0 := \mbox{span}\left( N_j^p, 0 \leq j \leq n  \right) 
%  \\
%  V_1 := \mbox{span}\left( M_j^{p-1}, 0 \leq j \leq n \right) 
%%  \label{}
%\end{align}
%We also define the operators $\pi_0$ and $\pi_1$ as the interpolation and histopolation operators on $V_0$ and $V_1$ respectively.
%We have the following result
%%
%\begin{lemma}
%  $\forall u \in H^s(\Omega)$, we have $~~ \diff \pi_0(u) = \pi_1 \left( \diff u \right) $
%%  \label{}
%\end{lemma}
\noindent
A special case, which is of interest for our Auxiliary Space Preconditioning method, is when the function lives in $\mathcal{S}_p(T)$ and the histopolation is done on $\mathcal{S}_{p-1}(T)$ using M-Splines. More precisely, we are interested in the case where Homogeneous Dirichlet boundary conditions are imposed and the interpolating degrees of freedom are removed. 
\noindent
Let $u := \sum\limits_{1 \le j \le n-1} u_j \Njone$, we have
\begin{align}
  \int_{x_i}^{x_{i+1}} u ~dx = \sum\limits_{1 \le j \le n} u_j \int_{x_i}^{x_{i+1}} \Njone ~dx \quad \quad 1 \leq i \leq n-1
\end{align}
which can be written in a matrix form as
\begin{align*}
  \begin{bmatrix}
    \int_{x_0}^{x_{1}}u ~dx\\
    \int_{x_1}^{x_{2}}u ~dx\\
    \vdots    \\ 
    \int_{x_n}^{x_{n+1}}u ~dx\\
  \end{bmatrix}
  &=
  \begin{bmatrix}
    \int_{x_0}^{x_{1}}N_1^p ~dx   & \ldots & \int_{x_0}^{x_{1}}N_{n-1}^p ~dx   \\
    \int_{x_1}^{x_{2}}N_1^p ~dx   & \ldots & \int_{x_1}^{x_{2}}N_{n-1}^p ~dx   \\
    \vdots                        & \ldots &                        \vdots \\
    \int_{x_n}^{x_{n+1}}N_1^p ~dx & \ldots & \int_{x_n}^{x_{n+1}}N_{n-1}^p ~dx \\
  \end{bmatrix}
  \begin{bmatrix}
    u_1\\
    u_2\\
    \vdots\\ 
    u_{n-1}\\
  \end{bmatrix}
  \\
  &= 
  \begin{bmatrix}
    \int_{x_0}^{x_{1}}M_1^{p-1} ~dx   & \ldots & \int_{x_0}^{x_{1}}M_{n-1}^{p-1} ~dx   \\
    \int_{x_1}^{x_{2}}M_1^{p-1} ~dx   & \ldots & \int_{x_1}^{x_{2}}M_{n-1}^{p-1} ~dx   \\
    \vdots                        & \ldots &                        \vdots \\
    \int_{x_n}^{x_{n+1}}M_1^{p-1} ~dx & \ldots & \int_{x_n}^{x_{n+1}}M_{n-1}^{p-1} ~dx \\
  \end{bmatrix}
  \begin{bmatrix}
    u_1^\star\\
    u_2^\star\\
    \vdots\\ 
    u_{n-1}^\star\\
  \end{bmatrix}
%  \label{}
\end{align*}
Finaly, we get
\begin{align}
  U^\star = \left( H^{M} \right)^{-1} H^{B} U  
%  \label{}
\end{align}
with
\begin{align}
  H^{M} &:= 
  \begin{bmatrix}
    \int_{x_0}^{x_{1}}M_1^{p-1} ~dx   & \ldots & \int_{x_0}^{x_{1}}M_{n-1}^{p-1} ~dx   \\
    \int_{x_1}^{x_{2}}M_1^{p-1} ~dx   & \ldots & \int_{x_1}^{x_{2}}M_{n-1}^{p-1} ~dx   \\
    \vdots                        & \ldots &                        \vdots \\
    \int_{x_n}^{x_{n+1}}M_1^{p-1} ~dx & \ldots & \int_{x_n}^{x_{n+1}}M_{n-1}^{p-1} ~dx \\
  \end{bmatrix}
  \\
  H^{B} &:= 
  \begin{bmatrix}
    \int_{x_0}^{x_{1}}N_1^p ~dx   & \ldots & \int_{x_0}^{x_{1}}N_{n-1}^p ~dx   \\
    \int_{x_1}^{x_{2}}N_1^p ~dx   & \ldots & \int_{x_1}^{x_{2}}N_{n-1}^p ~dx   \\
    \vdots                        & \ldots &                        \vdots \\
    \int_{x_n}^{x_{n+1}}N_1^p ~dx & \ldots & \int_{x_n}^{x_{n+1}}N_{n-1}^p ~dx \\
  \end{bmatrix}
  \label{eq:histopolation-h10}
\end{align}

\section{Kronecker algebra}
\label{sec:produit_kronecker_sec}
In this section, we present an overview about an interesting subject, which is the Kronecker Algebra, and which will be of a big interest in the Fast-IGA approach. More details about Kronecker Algebra can be found in \cite{vanloan2000,Graham_book,Bernstein_book}.

\begin{definition}[The $\mathbf{vec}$ operator]
Let $A=(a_{ij}) \in \mathcal{M}_{n \times m}$, the $\mathbf{vec}$ operator is defined as,
\begin{align}
\mathbf{vec} A = \left(\begin{array}{c}
 A_{:,1}
\\
\vdots
\\
 A_{:,m}
\end{array}\right) \in \mathbb{R}^{mn}
\end{align}
which is simply a vector composed by stacking all the columns of $A$. Where we denote $ A_{:,j}$ the $j^{th}$ column of $A$.
\\
We also define the inverse operator of $\mathbf{vec}$ by,
\begin{align}
A = \mathbf{vec}^{-1} \mathbf{vec} A
\end{align}
\end{definition}

\begin{definition}[Kronecker product]
Let $A=(a_{ij}) \in \mathcal{M}_{m \times n}$ and $B=(b_{ij}) \in \mathcal{M}_{r \times s}$ be two matrices. The Kronecker product of $A$ and $B$, denoted by $A \otimes B  \in \mathcal{M}_{mr \times ns}$, defines the following matrix:
\begin{align}
A \otimes B = 
\left(\begin{array}{cccc}
a_{11}B & a_{12}B & \cdots & a_{1n}B 
\\
a_{21}B & a_{22}B & \cdots & a_{2n}B  
\\
\vdots & \vdots &  & \vdots 
\\
a_{m1}B & a_{m2}B & \cdots & a_{mn}B 
\end{array}\right)
\end{align}
\end{definition}

\subsubsection*{Example}
Let 
\begin{align*}
A = 
\left(\begin{array}{cc}
a_{11} & a_{12}
\\
a_{21} & a_{22}
\end{array}\right),~~~
B = 
\left(\begin{array}{cc}
b_{11} & b_{12}
\\
b_{21} & b_{22}
\end{array}\right)
\end{align*}
then their Kronecker product is,
\begin{align}
A \otimes B = 
\left(\begin{array}{cccc}
a_{11}b_{11} & a_{11}b_{12} & a_{12}b_{11} & a_{12}b_{12}
\\
a_{11}b_{21} & a_{11}b_{22} & a_{12}b_{21} & a_{12}b_{22}
\\
a_{21}b_{11} & a_{21}b_{12} & a_{22}b_{11} & a_{22}b_{12}
\\
a_{21}b_{21} & a_{21}b_{22} & a_{22}b_{21} & a_{22}b_{22}
\end{array}\right)
\end{align}


\subsubsection*{Properties}
In the sequel, we recall some basic properties of the Kronecker product.
\begin{proposition}
  We have the following properties
  \begin{enumerate}[label=(\alph*).]
    \item If $\alpha$ is a scalar, then 
      \begin{equation}
      A \otimes \alpha B = \alpha A \otimes B
      \end{equation}
    \item Distributivity 
      \begin{equation}
        \begin{cases}
          ( A + B ) \otimes C = A \otimes C + B \otimes C \\
          A \otimes ( B + C ) = A \otimes B + A \otimes C
        \end{cases}
      \end{equation}
    \item Associativity      
      \begin{equation}
      A \otimes B \otimes C = A \otimes ( B \otimes C ) = ( A \otimes B ) \otimes C
      \end{equation}
    \item  Mixed Product Rule 
      \begin{equation}
      ( A \otimes B ) ( C \otimes D ) = AC \otimes BD
      \end{equation}
%      and,
%      \begin{equation}
%      ( A \otimes B ) ^p = A^p \otimes B^p,~~~\forall p \in \mathbb{N}
%      \end{equation}
    \item Transposition      
      \begin{equation}
      ( A \otimes B )^T = A^T \otimes B^T
      \end{equation}
    \item Inverse      
      \begin{equation}
      ( A \otimes B )^{-1} = A^{-1} \otimes B^{-1}
      \end{equation}
    \item The $\mathbf{vec}$ operator      
      \begin{equation}
      \label{kronecker_vec_abc}
      \mathbf{vec} ( ABC ) = ( C^T \otimes A ) \mathbf{vec} ( B )
      \end{equation}
    \item Trace      
      \begin{equation}
      \mathbf{tr} ( A \otimes B ) = \mathbf{tr} ( B \otimes A ) =  \mathbf{tr} ( A ) \mathbf{tr} ( B )
      \end{equation}
    \item       
      Let $A \in \mathcal{M}_{n \times n}$ and $B \in \mathcal{M}_{m \times m}$, we have,
      \begin{equation}
      \label{kronecker_prod_spec}
      \mathbf{mspec} ( A \otimes B ) = \{ \lambda \mu,~~\lambda \in \mathbf{mspec}(A),~\mu \in \mathbf{mspec}(B) \}
      \end{equation}
    \item  
      Let $A \in \mathcal{M}_{n \times n}$ and $B \in \mathcal{M}_{m \times m}$, then we have the following properties,
      \begin{itemize}
      \item if $A$ and $B$ are diagonal, then $A\otimes B$ is diagonal,
      \item if $A$ and $B$ are upper triangular, then $A\otimes B$ is upper triangular,
      \item if $A$ and $B$ are lower triangular, then $A\otimes B$ is lower triangular,
      \end{itemize}
  \end{enumerate}
\end{proposition}

%\begin{proposition}
%If $\alpha$ is a scalar, then 
%\begin{align}
%A \otimes \alpha B = \alpha A \otimes B
%\end{align}
%\end{proposition}

%\begin{proposition}
%We have,
%\begin{align}
%( A + B ) \otimes C &= A \otimes C + B \otimes C
%\\
%A \otimes ( B + C ) &= A \otimes B + A \otimes C
%\end{align}
%\end{proposition}

%\begin{proposition}[Associativity]
%\begin{align}
%A \otimes B \otimes C = A \otimes ( B \otimes C ) = ( A \otimes B ) \otimes C
%\end{align}
%\end{proposition}

%\begin{proposition}[Mixed Product Rule]
%\begin{align}
%( A \otimes B ) ( C \otimes D ) = AC \otimes BD
%\end{align}
%and,
%\begin{align}
%( A \otimes B ) ^p = A^p \otimes B^p,~~~\forall p \in \mathbb{N}
%\end{align}
%\end{proposition}

%\begin{proposition}
%\begin{align}
%( A \otimes B )^T = A^T \otimes B^T
%\end{align}
%\end{proposition}

%\begin{proposition}
%\begin{align}
%( A \otimes B )^{-1} = A^{-1} \otimes B^{-1}
%\end{align}
%\end{proposition}

%\begin{proposition}
%\begin{align}
%\label{kronecker_vec_abc}
%\mathbf{vec} ( ABC ) = ( C^T \otimes A ) \mathbf{vec} ( B )
%\end{align}
%\end{proposition}

%\begin{proposition}
%\begin{align}
%\mathbf{tr} ( A \otimes B ) = \mathbf{tr} ( B \otimes A ) =  \mathbf{tr} ( A ) \mathbf{tr} ( B )
%\end{align}
%\end{proposition}

%\begin{proposition}
%Let $A \in \mathcal{M}_{n \times n}$ and $B \in \mathcal{M}_{m \times m}$, we have,
%\begin{align}
%\label{kronecker_prod_spec}
%\mathbf{mspec} ( A \otimes B ) = \{ \lambda \mu,~~\lambda \in \mathbf{mspec}(A),~\mu \in \mathbf{mspec}(B) \}
%\end{align}
%\end{proposition}

%\begin{proposition}
%Let $A \in \mathcal{M}_{n \times n}$ and $B \in \mathcal{M}_{m \times m}$, we have,
%\begin{align}
%\mathbf{det} ( A \otimes B ) = ( \mathbf{det} A )^m ( \mathbf{det} B )^n
%\end{align}
%\end{proposition}
%We deduce from \ref{kronecker_prod_spec},
%\begin{proposition}
%Let $A \in \mathcal{M}_{n \times n}$, we have,
%\begin{align}
%\rho ( A \otimes A ) = \rho ( A )^2
%\end{align}
%\end{proposition}

%\begin{proposition}
%Let $f$ be an analytic function, $A \in \mathcal{M}_{n \times n}$ such that $f(A)$ exists, then we have,
%\begin{align}
%f(I_m \otimes A) = I_m \otimes f(A)
%\\
%f( A \otimes I_m ) = f(A) \otimes I_m 
%\end{align}
%\end{proposition}

%\begin{proposition}
%Let $X \in \mathbb{R}^{n}$ and $Y \in \mathbb{R}^{m}$, be two vectors. We have,
%\begin{align}
%X Y^T = X \otimes (Y^T) = (Y^T) \otimes X
%\end{align}
%moreover, we have,
%\begin{align}
%\mathbf{vec} (XY^T) = Y \otimes X 
%\end{align}
%\end{proposition}
%
%\begin{definition}[Kronecker permutation matrix]
%The Kronecker permutation matrix $P_{n,m} \in  \mathcal{M}_{nm \times nm}$, is defined by,
%\begin{align}
%P_{n,m} = \sum_{i,j=1}^{n,m} E_{i,j,n \times m} \otimes E_{j,i,m \times n}
%\end{align}
%\end{definition}

%\begin{proposition}
%Let $A \in \mathcal{M}_{m \times n}$, we have,
%\begin{align}
%\mathbf{vec} (A^T) = P_{m,n} \mathbf{vec} (A) 
%\end{align}
%\end{proposition}
%
%\begin{proposition}
%Let us consider the Kronecker permutation matrix $P_{n,m} \in  \mathcal{M}_{nm \times nm}$. Then we have,
%\begin{itemize}
%\item $P_{n,m}^T=P_{n,m}^{-1}=P_{m,n}$
%\item $P_{n,m}$ is orthogonal,
%\item $P_{n,m} P_{m,n} =I_{nm}$
%\item $P_{n,n}$ is orthogonal, symmetric and involutory,
%\item $P_{n,n}$ is a reflector,
%\item $\mathbf{tr} P_{n,n} = n$,
%\item $P_{1,m}=I_{m}$, and $P_{n,1}=I_{n}$
%\item if $X \in \mathbb{R}^{n}$ and $Y \in \mathbb{R}^{m}$, then,
%\begin{align}
%P_{n,m} ( Y \otimes X ) = X \otimes Y
%\end{align}
%\item if $A \in \mathcal{M}_{n \times m}$ and $B \in \mathcal{M}_{r \times s}$, then
%\begin{align}
%P_{r,n} ( A \otimes B ) P_{m,s} = B \otimes A
%\end{align}
%\item if $A \in \mathcal{M}_{n \times n}$ and $B \in \mathcal{M}_{m \times m}$, then
%\begin{align}
%P_{m,n} ( A \otimes B ) P_{n,m} =  P_{m,n} ( A \otimes B ) P_{m,n}^{-1} = B \otimes A
%\end{align}
%Therefor, $A \otimes B$ and $B \otimes A$ are similar.
%\end{itemize}
%\end{proposition}

%\begin{proposition}
%Let $A \in \mathcal{M}_{n \times n}$ and $B \in \mathcal{M}_{m \times m}$, then we have the following properties,
%\begin{itemize}
%\item if $A$ and $B$ are diagonal, then $A\otimes B$ is diagonal,
%\item if $A$ and $B$ are upper triangular, then $A\otimes B$ is upper triangular,
%\item if $A$ and $B$ are lower triangular, then $A\otimes B$ is lower triangular,
%\end{itemize}
%\end{proposition}

%\begin{proposition}
%Let $A,C \in \mathcal{M}_{n \times m}$ and $B,D \in \mathcal{M}_{r \times s}$. If $A$ is (left equivalent, right equivalent, equivalent) to $C$, and assume that $B$ is (left equivalent, right equivalent, equivalent) to $D$. Then, $A \otimes B$ is (left equivalent, right equivalent, equivalent) to $C \otimes D$.
%\end{proposition}

%\begin{remark}
%The use of Kronecker product preconditioners is well known \cite{vanloan,Langville_Stewart,Elisabeth_Ullmann,GRIGORI:2008:INRIA-00268301:5}, it is based on results of the form,
%\begin{align} 
%\mbox{Minimizing,}
%~~~~~~~~
%\phi_A(B,C) = \| A - B \otimes C  \|^2
%\end{align}
%for a chosen norm.
%\end{remark}

%\subsection{Kronecker sum}
%\begin{definition}[Kronecker sum]
%Let $A=(a_{ij}) \in \mathcal{M}_{n \times n}$ and $B=(b_{ij}) \in \mathcal{M}_{m \times m}$ be two matrices. The Kronecker sum of $A$ and $B$, denoted by $A \oplus B  \in \mathcal{M}_{mn \times mn}$, defines the following matrix:
%\begin{align}
%A \oplus B = A \otimes I_m + I_n \otimes B
%\end{align}
%\end{definition}
%
%\begin{proposition}
%Let $A \in \mathcal{M}_{n \times n}$ and $B \in \mathcal{M}_{m \times m}$, we have,
%\begin{align}
%\label{kronecker_sum_spec}
%\mathbf{mspec} ( A \oplus B ) = \{ \lambda +  \mu,~~\lambda \in \mathbf{mspec}(A),~\mu \in \mathbf{mspec}(B) \}
%\end{align}
%\end{proposition}
%
%\subsection{Solving $AX+XB=C$}
%Let $A \in \mathcal{M}_{n \times n}$, $B \in \mathcal{M}_{m \times m}$ and $C \in \mathcal{M}_{n \times m}$. The aim of this section, is to solve the equation:
%\begin{align}
%\label{kronecker_eq1}
%AX+XB=C
%\end{align}
%we can rewrite this equation in term of the Kronecker sum:
%\begin{align}
%(B^T \oplus A)\mathbf{vec}X =\mathbf{vec}C
%\end{align}
%or equivalently,
%\begin{align}
%G x = c
%\end{align}
%where,
%$G = (B^T \oplus A)$, $x = \mathbf{vec}X $, and $c = \mathbf{vec}C$.
%\\
%Using the property \ref{kronecker_sum_spec}, we can easily check that \ref{kronecker_eq1} has a unique solution if and only if $G$ is nonsingular, \textit{i.e} $\lambda +  \mu \neq 0,~~\forall \lambda \in \mathbf{mspec}(A),~\forall \mu \in \mathbf{mspec}(B)$, which can be written in the form,
%\begin{align}
%\mathbf{mspec}(A) \cap \mathbf{mspec}(-B) = \emptyset
%\end{align}
%
%\begin{proposition}
%If $\mathbf{mspec}(A) \cap \mathbf{mspec}(-B) = \emptyset$, then there exists a unique matrix $X \in \mathcal{M}_{n \times m}$, satisfying \ref{kronecker_eq1}. Moreover, the matrices $\left(\begin{array}{cc}
%A & C
%\\
%0 & -B
%\end{array}\right)$ and $\left(\begin{array}{cc}
%A & 0
%\\
%0 & -B
%\end{array}\right)$ are similar and verify,
%\begin{align}
%\left(\begin{array}{cc}
%A & C
%\\
%0 & -B
%\end{array}\right) = \left(\begin{array}{cc}
%I & X
%\\
%0 & I
%\end{array}\right)
%\left(\begin{array}{cc}
%A & 0
%\\
%0 & -B
%\end{array}\right)
%\left(\begin{array}{cc}
%I & -X
%\\
%0 & I
%\end{array}\right).
%\end{align}
%\end{proposition}
%
%\subsection{Solving $AXB=C$}
%Let $A,B,C$ and $X \in \mathcal{M}_{n \times n}$. As seen previously, using \ref{kronecker_vec_abc}, the equation 
%\begin{align}
%\label{kronecker_eq2}
%AXB=C
%\end{align}
%can be written in the form,
%\begin{align}
%H x = c
%\end{align}
%where,
%$H = (B^T \otimes A)$, $x = \mathbf{vec}X $, and $c = \mathbf{vec}C$.
%\\
%Using the property \ref{kronecker_prod_spec}, we can easily check that \ref{kronecker_eq2} has a unique solution if and only if $H$ is nonsingular, \textit{i.e} $\lambda  \mu \neq 0,~~\forall \lambda \in \mathbf{mspec}(A),~\forall \mu \in \mathbf{mspec}(B)$, which is equivalent to, $A$ and $B$ are both nonsingular.
%
%%\subsection{Solving $\sum_{i=1}^r A_i X B_i=C$}
%%Let $A_i,B_i,C,~~1 \leq i \leq r$ and $X \in \mathcal{M}_{n \times n}$. Using, the previous result, it is easy to show that the solution of:
%%\begin{align}
%%\label{kronecker_eq3}
%%\sum_{i=1}^r A_i X B_i=C
%%\end{align}
%%can be written in the form,
%%\begin{align}
%%H x = c
%%\end{align}
%%where,
%%$H = \sum_{i=1}^r (B_i^T \otimes A_i)$, $x = \mathbf{vec}X $, and $c = \mathbf{vec}C$.


\clearpage
\section{Matrix form of the variational problem}
In the sequel, we derive the matrix form related to the discrete variational formulation of \eqref{eq:curl-variational-problem}. 
We shall consider a computational domain $\Omega$ as unit square $(0,1)^2$ or unit cube $(0,1)^3$ and derive the matrix form using the Kronecker Algebra. 
Before expliciting the matrix forms in 2D and 3D, we start by introducing some 1D matrices that we will need.
\begin{align}
  \begin{cases}
    \left( M_{s} \right)_{i_s, j_s} &= \int_{0}^1 \Nis \Njs \dd x_s
    \\
    \left( K_{s} \right)_{i_s, j_s} &= \int_{0}^1 \Nis^\prime \Njs^\prime \dd x_s
    \\
    \left( D_{s} \right)_{i_s, j_s} &= \int_{0}^1 \Mis \Mjs \dd x_s
    \\
    \left( R_{s} \right)_{i_s, j_s} &= \int_{0}^1 \Nis^\prime \Mjs \dd x_s
  \end{cases}
\end{align}

\subsection{2D case}
We consider the discrete variational formulation of \eqref{eq:curl-variational-problem}. 
For the sake of simplicity we shall introduce the scalar functions
\begin{align*}
  \Psi^1_{\jj} = \Mjone \Njtwo 
  \\
  \Psi^2_{\jj} = \Njone \Mjtwo  
\end{align*}
we also define the vectors $\ee_1 = \begin{bmatrix} 1 \\ 0 \end{bmatrix}$ and $\ee_2 = \begin{bmatrix} 0 \\ 1 \end{bmatrix}$. Therefor, the expression of $\uu_h \in \Vcurl$ becomes
\begin{align*}
  \uu_h = \sum\limits_{\jj} \left( 
    u_{\jj}^1 \Psi^1_{\jj} \ee_1 
  + u_{\jj}^2 \Psi^2_{\jj} \ee_2 
  \right)
\end{align*}
we find that $A$ is a symmetric $2\times 2$ block matrix of the form
\begin{align}
  A = 
  \begin{bmatrix}
    A_{11}   & A_{12} \\
    A_{12}^T & A_{22} \\
  \end{bmatrix}
%  \label{}
\end{align}
where
\begin{align*}
  {A_{11}}_{\ii, \jj} &=   \int_{\Omega} 
         \partial_{x_2} \Psi^{1}_{\jj} \partial_{x_2} \Psi^{1}_{\ii}
        \dd \mathbf{x} 
                       +   \int_{\Omega} 
                       \tau \Psi^{1}_{\jj} \Psi^{1}_{\ii} 
        \dd \mathbf{x} 
  \\
  {A_{12}}_{\ii, \jj} &= - \int_{\Omega} 
          \partial_{x_2} \Psi^{1}_{\jj} \partial_{x_1} \Psi^{2}_{\ii}
        \dd \mathbf{x} 
  \\
  {A_{22}}_{\ii, \jj} &=   \int_{\Omega} 
        \partial_{x_1} \Psi^{2}_{\jj} ~ \partial_{x_1} \Psi^{2}_{\ii}
        \dd \mathbf{x} 
                       +   \int_{\Omega} 
                       \tau \Psi^{2}_{\jj} \Psi^{2}_{\ii} 
        \dd \mathbf{x} 
\end{align*}
For the right hand side, the entries associated to each component of the vector $\ff$ are given by
\begin{align*}
    F_{1, i} & = \int_{\Omega} \ff_1 \Psi^1_{i} \dd \mathbf{x} \\
    F_{2, i} & = \int_{\Omega} \ff_2 \Psi^2_{i} \dd \mathbf{x} \\
\end{align*}
Hence, we have,
\begin{align*}
  \begin{cases}
    {A_{11}}_{\ii, \jj} &= \left( D_1 \otimes K_2 \right)_{\ii \jj} 
                         + \tau \left( D_1 \otimes M_2 \right)_{\ii \jj} 
    \\
    {A_{12}}_{\ii, \jj} &= - \left( R_1 \otimes R_2^T \right)_{\ii \jj} 
    \\
    {A_{22}}_{\ii, \jj} &= \left( K_1 \otimes D_2 \right)_{\ii \jj} 
                         + \tau \left( M_1 \otimes D_2 \right)_{\ii \jj} 
  \end{cases}
\end{align*}
Therefor, we have the following matrix form 
\begin{align}
  \mathcal{A}^\tau = \mathcal{A}^0 + \tau \mathcal{M}
%  \label{}
\end{align}
where
\begin{align}
  \mathcal{A}^0 = 
  \begin{bmatrix}
        D_1 \otimes K_2  & - R_1 \otimes R_2^T  \\
    - R_1^T \otimes R_2  &   K_1 \otimes D_2 
  \end{bmatrix}
  \label{eq:matrix-A0-2d}
\end{align}
and
\begin{align}
  \mathcal{M} = 
  \begin{bmatrix}
    D_1 \otimes M_2   & 0  \\
    0 & M_1 \otimes D_2   
  \end{bmatrix}
  \label{eq:matrix-M-2d}
\end{align}

\subsection{3D case}
We consider the discrete variational formulation of \eqref{eq:curl-variational-problem}. 
For the sake of simplicity we shall introduce the scalar functions
\begin{align*}
  \Psi^1_{\jj} = \Mjone \Njtwo \Njtre 
  \\
  \Psi^2_{\jj} = \Njone \Mjtwo \Njtre  
  \\
  \Psi^3_{\jj} = \Njone \Njtwo \Mjtre  
\end{align*}
we also define the vectors $\ee_1 = \begin{bmatrix} 1 \\ 0 \\ 0 \end{bmatrix}$, $\ee_2 = \begin{bmatrix} 0 \\ 1 \\ 0 \end{bmatrix}$ and $\ee_3 = \begin{bmatrix} 0 \\ 0 \\ 1 \end{bmatrix}$. Therefor, the expression of $\uu_h \in \Vcurl$ becomes
\begin{align*}
  \uu_h = \sum\limits_{\jj} \left( 
    u_{\jj}^1 \Psi^1_{\jj} \ee_1 
  + u_{\jj}^2 \Psi^2_{\jj} \ee_2 
  + u_{\jj}^3 \Psi^3_{\jj} \ee_3 
  \right)
\end{align*}
On the other hand, we have,
\begin{align*}
 \Curl \Psi^{1}_{\jj} \ee_{1} &= 
   \partial_{x_3} \Psi^{1}_{\jj} \ee_2 
 - \partial_{x_2} \Psi^{1}_{\jj} \ee_3
 \\
 \Curl \Psi^{2}_{\jj} \ee_{2} &= 
 - \partial_{x_3} \Psi^{2}_{\jj} \ee_1
 + \partial_{x_1} \Psi^{2}_{\jj} \ee_3
 \\
 \Curl \Psi^{3}_{\jj} \ee_{3} &= 
   \partial_{x_2} \Psi^{3}_{\jj} \ee_1 
 - \partial_{x_1} \Psi^{3}_{\jj} \ee_2
\end{align*}
Because $\ee_i \cdot \ee_j = \delta_{ij}$, we get, 
\begin{align*}
 \Curl \Psi^{1}_{\jj} \ee_{1} \cdot \Curl \Psi^{1}_{\ii} \ee_{1} &=  
    \partial_{x_3} \Psi^{1}_{\jj} ~ \partial_{x_3} \Psi^{1}_{\ii} 
  + \partial_{x_2} \Psi^{1}_{\jj} ~ \partial_{x_2} \Psi^{1}_{\ii}
 \\
 \Curl \Psi^{1}_{\jj} \ee_{1} \cdot \Curl \Psi^{2}_{\ii} \ee_{2} &= 
 - \partial_{x_2} \Psi^{1}_{\jj} ~ \partial_{x_1} \Psi^{2}_{\ii}
 \\
 \Curl \Psi^{1}_{\jj} \ee_{1} \cdot \Curl \Psi^{3}_{\ii} \ee_{3} &=  
 - \partial_{x_3} \Psi^{1}_{\jj} ~ \partial_{x_1} \Psi^{3}_{\ii}
 \\
% \Curl \Psi^{2}_{\jj} \ee_{2} \cdot \Curl \Psi^{1}_{\ii} \ee_{1} &=  
% - \partial_{x_1} \Psi^{2}_{\jj} ~ \partial_{x_2} \Psi^{1}_{\jj}
% \\
 \Curl \Psi^{2}_{\jj} \ee_{2} \cdot \Curl \Psi^{2}_{\ii} \ee_{2} &= 
   \partial_{x_3} \Psi^{2}_{\jj} ~ \partial_{x_3} \Psi^{2}_{\ii} 
 + \partial_{x_1} \Psi^{2}_{\jj} ~ \partial_{x_1} \Psi^{2}_{\ii}
 \\
 \Curl \Psi^{2}_{\jj} \ee_{2} \cdot \Curl \Psi^{3}_{\ii} \ee_{3} &=  
 - \partial_{x_3} \Psi^{2}_{\jj} ~  \partial_{x_2} \Psi^{3}_{\ii} 
 \\
% \Curl \Psi^{3}_{\jj} \ee_{3} \cdot \Curl \Psi^{1}_{\ii} \ee_{1} &=  
% - \partial_{x_1} \Psi^{3}_{\jj} ~ \partial_{x_3} \Psi^{1}_{\ii} 
% \\
% \Curl \Psi^{3}_{\jj} \ee_{3} \cdot \Curl \Psi^{2}_{\ii} \ee_{2} &= 
% - \partial_{x_2} \Psi^{3}_{\jj} ~ \partial_{x_3} \Psi^{2}_{\ii} 
% \\
 \Curl \Psi^{3}_{\jj} \ee_{3} \cdot \Curl \Psi^{3}_{\ii} \ee_{3} &=  
   \partial_{x_2} \Psi^{3}_{\jj} ~ \partial_{x_2} \Psi^{3}_{\ii} 
 + \partial_{x_1} \Psi^{3}_{\jj} ~ \partial_{x_1} \Psi^{3}_{\ii}
\end{align*}
we find that $A$ is a symmetric $3\times 3$ block matrix of the form
\begin{align}
  A = 
  \begin{bmatrix}
    A_{11}   & A_{12}   &  A_{13} \\
    A_{12}^T & A_{22}   &  A_{23} \\
    A_{13}^T & A_{23}^T &  A_{33} 
  \end{bmatrix}
%  \label{}
\end{align}
where
\begin{align*}
  {A_{11}}_{\ii, \jj} &=   \int_{\Omega} 
          \partial_{x_3} \Psi^{1}_{\jj} \partial_{x_3} \Psi^{1}_{\ii} 
        + \partial_{x_2} \Psi^{1}_{\jj} \partial_{x_2} \Psi^{1}_{\ii}
        \dd \mathbf{x} 
                       +   \int_{\Omega} 
                       \tau \Psi^{1}_{\jj} \Psi^{1}_{\ii} 
        \dd \mathbf{x} 
  \\
  {A_{12}}_{\ii, \jj} &= - \int_{\Omega} 
          \partial_{x_2} \Psi^{1}_{\jj} \partial_{x_1} \Psi^{2}_{\ii}
        \dd \mathbf{x} 
  \\
  {A_{13}}_{\ii, \jj} &= -  \int_{\Omega} 
         \partial_{x_3} \Psi^{1}_{\jj} \partial_{x_1} \Psi^{3}_{\ii}
        \dd \mathbf{x} 
  \\
%  {A_{21}}_{\ii, \jj} &= -   \int_{\Omega} 
%        \partial_{x_1} \Psi^{2}_{\jj} ~ \partial_{x_2} \Psi^{1}_{\jj}
%        \dd \mathbf{x} 
%  \\
  {A_{22}}_{\ii, \jj} &=   \int_{\Omega} 
         \partial_{x_3} \Psi^{2}_{\jj} ~ \partial_{x_3} \Psi^{2}_{\ii} 
       + \partial_{x_1} \Psi^{2}_{\jj} ~ \partial_{x_1} \Psi^{2}_{\ii}
        \dd \mathbf{x} 
                       +   \int_{\Omega} 
                       \tau \Psi^{2}_{\jj} \Psi^{2}_{\ii} 
        \dd \mathbf{x} 
  \\
  {A_{23}}_{\ii, \jj} &=  -  \int_{\Omega} 
       \partial_{x_3} \Psi^{2}_{\jj} ~  \partial_{x_2} \Psi^{3}_{\ii} 
        \dd \mathbf{x} 
  \\
%  {A_{31}}_{\ii, \jj} &= -   \int_{\Omega} 
%        \partial_{x_1} \Psi^{3}_{\jj} ~ \partial_{x_3} \Psi^{1}_{\ii} 
%        \dd \mathbf{x} 
%  \\
%  {A_{32}}_{\ii, \jj} &= -   \int_{\Omega} 
%       \partial_{x_2} \Psi^{3}_{\jj} ~ \partial_{x_3} \Psi^{2}_{\ii} 
%        \dd \mathbf{x} 
%  \\
  {A_{33}}_{\ii, \jj} &=   \int_{\Omega} 
         \partial_{x_2} \Psi^{3}_{\jj} ~ \partial_{x_2} \Psi^{3}_{\ii} 
       + \partial_{x_1} \Psi^{3}_{\jj} ~ \partial_{x_1} \Psi^{3}_{\ii}
        \dd \mathbf{x} 
                       +   \int_{\Omega} 
                       \tau \Psi^{3}_{\jj} \Psi^{3}_{\ii} 
        \dd \mathbf{x} 
\end{align*}
For the right hand side, the entries associated to each component of the vector $\ff$ are given by
\begin{align*}
    F_{1, i} & = \int_{\Omega} \ff_1 \Psi^1_{i} \dd \mathbf{x} \\
    F_{2, i} & = \int_{\Omega} \ff_2 \Psi^2_{i} \dd \mathbf{x} \\
    F_{3, i} & = \int_{\Omega} \ff_3 \Psi^3_{i} \dd \mathbf{x} 
\end{align*}
Hence, we have,
\begin{align*}
  \begin{cases}
    {A_{11}}_{\ii, \jj} &= \left( D_1 \otimes M_2 \otimes K_3 \right)_{\ii \jj} 
                         + \left( D_1 \otimes K_2 \otimes M_3 \right)_{\ii \jj} 
                         + \tau \left( D_1 \otimes M_2 \otimes M_3 \right)_{\ii \jj} 
    \\
    {A_{12}}_{\ii, \jj} &= - \left( R_1 \otimes R_2^T \otimes M_3 \right)_{\ii \jj} 
    \\
    {A_{13}}_{\ii, \jj} &= - \left( R_1 \otimes M_2 \otimes R_3^T \right)_{\ii \jj} 
    \\
    {A_{22}}_{\ii, \jj} & = \left( M_1 \otimes D_2 \otimes K_3 \right)_{\ii \jj} 
                         + \left( K_1 \otimes D_2 \otimes M_3 \right)_{\ii \jj} 
                         + \tau \left( M_1 \otimes D_2 \otimes M_3 \right)_{\ii \jj} 
    \\
    {A_{23}}_{\ii, \jj} &= - \left( M_1 \otimes R_2 \otimes R_3^T \right)_{\ii \jj} 
    \\
    {A_{33}}_{\ii, \jj} &= \left( M_1 \otimes K_2 \otimes D_3 \right)_{\ii \jj} 
                         + \left( K_1 \otimes M_2 \otimes D_3 \right)_{\ii \jj} 
                         + \tau \left( M_1 \otimes M_2 \otimes D_3 \right)_{\ii \jj} 
  \end{cases}
\end{align*}
Therefor, we have the following matrix form 
\begin{align}
  \mathcal{A}^\tau = \mathcal{A}^0 + \tau \mathcal{M}
%  \label{}
\end{align}
where
\begin{align}
  \mathcal{A}^0 = 
  \begin{bmatrix}
    D_1 \otimes M_2 \otimes K_3 + D_1 \otimes K_2 \otimes M_3  &                          - R_1 \otimes R_2^T \otimes M_3   &  - R_1 \otimes M_2 \otimes R_3^T \\
                               - R_1^T \otimes R_2 \otimes M_3 & M_1 \otimes D_2 \otimes K_3 + K_1 \otimes D_2 \otimes M_3  &  - M_1 \otimes R_2 \otimes R_3^T \\
                               - R_1^T \otimes M_2 \otimes R_3 &                          - M_1 \otimes R_2^T \otimes R_3   & M_1 \otimes K_2 \otimes D_3 + K_1 \otimes M_2 \otimes D_3  
  \end{bmatrix}
  \label{eq:matrix-A0-3d}
\end{align}
and
\begin{align}
  \mathcal{M} = 
  \begin{bmatrix}
    D_1 \otimes M_2 \otimes M_3   & 0   &  0 \\
    0 & M_1 \otimes D_2 \otimes M_3   &  0 \\
    0 & 0 &  M_1 \otimes M_2 \otimes D_3 
  \end{bmatrix}
  \label{eq:matrix-M-3d}
\end{align}

\section{Auxiliary Space Preconditioners}\label{sec:asp}
The objective of this section is the construction of an appropriate auxiliary space preconditioner for our \textbf{curl}-\textbf{curl} problem. As already mentioned, the main challenge is the derivation of a discrete version of the regular decomposition of Theorem \ref{regular-decomposition-of-H(curl)}, namely the Hitmair-Xu decomposition. The presentation is in three subsections. First, in Subsection \ref{subsec:asp} we focus on the case without mapping where $\Omega$ is simply a parametric domain given by $\Omega=(0,1)^3$ and we show the discrete Hitmair-Xu decomposition in this case. The result of this subsection is extended to the case of a physical domain in Subsection \ref{subsec:asp-with-mapping}. The construction of the preconditioner is developed in Subsection \ref{subsec:construction-of-ASP}.



Through this section $A \lesssim B$ means that there exists some constant $C>0$, which is independent of $h$, such that $A \leq CB$.
 
\subsection{Discrete Decompositions without Mapping}\label{subsec:asp}
Throughout this subsection, $\Omega=(0,1)^3$.

We need the following preliminary results in order to prove Hitpmair-Xu decomposition stated in Proposition \ref{prop:Hitpmair-Xu decomposition}.

\begin{lemma}\label{prelimibary-lemma-1}
For every $\bm{\varphi} \in \mathbb{H}^1(\Omega)$ such that $\textbf{curl} \, (\bm{\varphi}) \in \bm{V}_h(div,\Omega)$, we have
\begin{itemize}
\item[(i)] $\Pi_h^{\textbf{curl}} \bm{\varphi}$ is well-defined;

\item[(ii)] $\textbf{curl}\,(\bm{\varphi})=\textbf{curl}\left( \Pi_h^{\textbf{curl}} \bm{\varphi}\right)$;

\item[(iii)] $\left\| \bm{\varphi}-\Pi_h^{\textbf{curl}} \bm{\varphi} \right\|_{\mathbb{L}^2(\Omega)} \lesssim h \|\bm{\varphi}\|_{\mathbb{H}^1(\Omega)}$.
\end{itemize}
\end{lemma}

\begin{proof}
First insertion is a consequence of the fact that $\mathbb{H}^1(\Omega) \subset \mathbf{H}(\textbf{curl},\Omega)$. Concerning (ii), using the commutativity of Diagram \eqref{pr-eq:DeRham-diagram}, we obtain
$$
\textbf{curl}\left( \Pi_h^{\textbf{curl}} \bm{\varphi}\right)= \Pi_h^{\text{div}} \left( \textbf{curl}\, (\bm{\varphi}) \right).
$$
We now use $\textbf{curl}\,(\bm{\varphi}) \in \bm{V}_h(div,\Omega)$ to obtain (ii). Estimate (iii) follows from Theorem \cite[Theorem 5.3]{buffa2011isogeometric} with the particular choice of $l=0$ and $s=1$. 
\end{proof}

\begin{lemma}\label{lem-semi-discrete-decomposition}
For each $\bm{u}_h \in \bm{V}_h(\textbf{curl},\Omega)$, there exist $\bm{\varphi} \in \mathbb{H}^1(\Omega)$ and $\phi_h \in V_h(\textbf{grad},\Omega)$ such that
\begin{equation}\label{eq:semi-discrete-decomposition}
\bm{u}_h = \Pi_h^{\textbf{curl}} \bm{\varphi} + \textbf{grad}\,(\phi_h),
\end{equation}
with estimates
\begin{equation}\label{eq:discrete-H-curl-estimate}
\|\bm{\varphi}\|_{\mathbb{H}^1(\Omega)} \lesssim \|\textbf{curl} \,(\bm{u}_h)\|_{\mathbb{L}^2(\Omega)},
\end{equation}
\begin{equation}\label{eq:stability-of-semi-discrete-decomposition}
\|\Pi_h^{\textbf{curl}} \bm{\varphi}\|_{\mathbb{L}^2(\Omega)} + \|\textbf{grad}\,(\phi_h)\|_{\mathbb{L}^2(\Omega)}\lesssim \|\bm{u}_h\|_{\mathbb{L}^2(\Omega)}.
\end{equation}
\end{lemma}

\begin{proof}
Let $\bm{u}_h \in \bm{V}_h(\textbf{curl},\Omega)$. According to Theorem \ref{regular-decomposition-of-H(curl)}, there exists $\bm{\varphi} \in \mathbb{H}^1(\Omega)$ such that
\begin{equation}\label{eq:semi-discrete-regular-decomposition}
\begin{cases}
\textbf{curl}\,(\bm{\varphi})=\textbf{curl}\,(\bm{u}_h) \in \bm{V}_h(div,\Omega)\\
\|\bm{\varphi}\|_{\mathbb{H}^1(\Omega)} \lesssim \|\textbf{curl}(\bm{u}_h)\|_{\mathbb{L}^2(\Omega)}\\
\|\bm{\varphi}\|_{\mathbb{L}^2(\Omega)} \lesssim \|\bm{u}_h\|_{\mathbb{L}^2(\Omega)}.
\end{cases}
\end{equation}
We now apply Lemma  \eqref{prelimibary-lemma-1} and obtain
$$
\textbf{curl}\,(\bm{u}_h)=\textbf{curl}\,(\bm{\varphi})=\textbf{curl}\left( \Pi_h^{\textbf{curl}} \bm{\varphi}\right),
$$
hence, 
$$\bm{u}_h - \Pi_h^{\textbf{curl}} \bm{\varphi} \in \textbf{ker}\left(\textbf{curl} \mid_{\bm{V}_h(\textbf{curl},\Omega)} \right)=\textbf{grad} \left( V_h(\textbf{grad},\Omega)\right).$$ 
Therefore, there exists $\phi_h \in V_h(\textbf{grad},\Omega)$  such that $\bm{u}_h - \Pi_h^{\textbf{curl}} \bm{\varphi}=\textbf{grad}\,(\phi_h)$,  which yields \eqref{eq:semi-discrete-decomposition}.  Estimate \eqref{eq:discrete-H-curl-estimate} follows from the second estimate in \eqref{eq:semi-discrete-regular-decomposition}.

We now show \eqref{eq:stability-of-semi-discrete-decomposition}. We write
\begin{eqnarray*}
\|\Pi_h^{\textbf{curl}} \bm{\varphi}\|_{\mathbb{L}^2(\Omega)} & \leq & \|\Pi_h^{\textbf{curl}} \bm{\varphi} -\bm{\varphi}\|_{\mathbb{L}^2(\Omega)}+\|\bm{\varphi}\|_{\mathbb{L}^2(\Omega)}\\
& \lesssim & h \|\bm{\varphi}\|_{\mathbb{H}^1(\Omega)} +  \|\bm{\varphi}\|_{\mathbb{L}^2(\Omega)}\\
& \lesssim &  h \|\textbf{curl}(\bm{u}_h)\|_{\mathbb{L}^2(\Omega)} + \|\bm{u}_h\|_{\mathbb{L}^2(\Omega)},
\end{eqnarray*}
where in the last estimate we have used first and second estimates in \eqref{eq:semi-discrete-regular-decomposition}. Moreover, using the inverse inequality 
\begin{equation}\label{eq:inverse-inequlity-curl}
\|\textbf{curl}(\bm{u}_h)\|_{\mathbb{L}^2(\Omega)} \lesssim h^{-1} \|\bm{u}_h\|_{\mathbb{L}^2(\Omega)},
\end{equation}
we get 
$$
\|\Pi_h^{\textbf{curl}} \bm{\varphi}\|_{\mathbb{L}^2(\Omega)} \lesssim \|\bm{u}_h\|_{\mathbb{L}^2(\Omega)}.
$$
On the other hand, we have 
\begin{eqnarray*}
\|\textbf{grad}\,(\phi_h)\|_{\mathbb{L}^2(\Omega)}  &=& \| \bm{u}_h - \Pi_h^{\textbf{curl}} \bm{\varphi}\|_{\mathbb{L}^2(\Omega)}\\
& \lesssim & \| \bm{u}_h \|_{\mathbb{L}^2(\Omega)} + \| \Pi_h^{\textbf{curl}} \bm{\varphi}\|_{\mathbb{L}^2(\Omega)} \lesssim \| \bm{u}_h \|_{\mathbb{L}^2(\Omega)},
\end{eqnarray*}
and  inequality \eqref{eq:stability-of-semi-discrete-decomposition} is proved.
\end{proof}

\begin{lemma}\label{lem-stable-approximation-of-vect-grad}
Every $\bm{\varphi} \in \mathbb{H}^1(\Omega)$ admits a stable approximation $\bm{\varphi}_h \in \mathbb{V}_h(\textbf{grad},\Omega)$ satisfying
$$
h^{-1} \|\bm{\varphi} -\bm{\varphi}_h \|_{\mathbb{L}^2(\Omega)} + \|\bm{\varphi}_h\|_{\mathbb{H}^1(\Omega)} \lesssim \|\bm{\varphi}\|_{\mathbb{H}^1(\Omega)}.
$$  
\end{lemma}

\begin{proof}
Let $\bm{\varphi}:=\left(\varphi^1,\varphi^2,\varphi^3\right) \in \mathbb{H}^1(\Omega)$ and define 
$$\bm{\varphi}_h= \left(\Pi_h^{\textbf{grad}} \varphi^1,\Pi_h^{\textbf{grad}} \varphi^2,\Pi_h^{\textbf{grad}} \varphi^3\right) \in \mathbb{V}_h(\textbf{grad},\Omega).$$
According to Theorem \cite[Theorem 5.3]{buffa2011isogeometric}, we have:
$$
\|\varphi^k - \Pi_h^{\textbf{grad}} \varphi^k\|_{L^2(\Omega)} \lesssim  h \|\varphi^k\|_{H^1(\Omega)}, \quad k=1,2,3 
$$
and
$$
\|\varphi^k - \Pi_h^{\textbf{grad}} \varphi^k\|_{H^1(\Omega)} \lesssim \|\varphi^k\|_{H^1(\Omega)}, \quad k=1,2,3. 
$$
Using these last two estimates, we get 
\begin{eqnarray}\label{eq:proof-stable-approximation-of-vect-grad-1}
\|\bm{\varphi} -\bm{\varphi}_h \|_{\mathbb{L}^2(\Omega)}^2 &=& \|\varphi^1 - \Pi_h^{\textbf{grad}} \varphi^1\|_{L^2(\Omega)}^2+\|\varphi^2 - \Pi_h^{\textbf{grad}} \varphi^2\|_{L^2(\Omega)}^2+\|\varphi^3 - \Pi_h^{\textbf{grad}} \varphi^3\|_{L^2(\Omega)}^2 \nonumber \\
& \lesssim & h^2 \left( \|\varphi^1\|_{H^1(\Omega)}^2 + \|\varphi^2\|_{H^1(\Omega)}^2 + \|\varphi^3\|_{H^1(\Omega)}^2 \right)= h^2 \|\bm{\varphi}\|_{\mathbb{H}^1(\Omega)}^2,
\end{eqnarray}
and
\begin{eqnarray*}
\|\bm{\varphi} -\bm{\varphi}_h \|_{\mathbb{H}^1(\Omega)}^2 &=& \|\varphi^1 - \Pi_h^{\textbf{grad}} \varphi^1\|_{H^1(\Omega)}^2+\|\varphi^2 - \Pi_h^{\textbf{grad}} \varphi^2\|_{H^1(\Omega)}^2+\|\varphi^3 - \Pi_h^{\textbf{grad}} \varphi^3\|_{H^1(\Omega)}^2\\
& \lesssim &  \|\varphi^1\|_{H^1(\Omega)}^2 + \|\varphi^2\|_{H^1(\Omega)}^2 + \|\varphi^3\|_{H^1(\Omega)}^2= \|\bm{\varphi}\|_{\mathbb{H}^1(\Omega)}^2,
\end{eqnarray*}
from which we deduce that 
\begin{equation}\label{eq:proof-stable-approximation-of-vect-grad-2}
\|\bm{\varphi}_h\|_{\mathbb{H}^1(\Omega)} \leq \|\bm{\varphi} -\bm{\varphi}_h \|_{\mathbb{H}^1(\Omega)} + \|\bm{\varphi}\|_{\mathbb{H}^1(\Omega)} \lesssim  \|\bm{\varphi}\|_{\mathbb{H}^1(\Omega)}.
\end{equation}
Combining \eqref{eq:proof-stable-approximation-of-vect-grad-1} and \eqref{eq:proof-stable-approximation-of-vect-grad-2} we conclude the proof.
\end{proof}

We have the following regular discrete decomposition.

\begin{proposition}[Hitpmair-Xu decomposition]\label{prop:Hitpmair-Xu decomposition}
Every $\bm{u}_h \in \bm{V}_h(\textbf{curl},\Omega)$ has a decomposition 
\begin{equation}\label{eq:Hitpmair-Xu-decomposition}
\bm{u}_h=\bm{w}_h+\Pi_h^{\textbf{curl}} \bm{\varphi}_h + \textbf{grad}\,(\phi_h),
\end{equation}
where $\bm{w}_h \in \bm{V}_h(\textbf{curl},\Omega)$, $\bm{\varphi}_h \in  \mathbb{V}_h(\textbf{grad},\Omega)$ and $\phi_h \in V_h(\textbf{grad},\Omega)$ with estimate
\begin{equation}\label{eq:Hitpmair-Xu-decomposition-estimate}
(h^{-2} + \mu) \, \|\bm{w}_h\|_{\mathbb{L}^2(\Omega)}^2 + \|\bm{\varphi}_h\|_{\mathbb{H}^1(\Omega)}^2 + \mu \|\bm{\varphi}_h\|_{\mathbb{L}^2(\Omega)}^2 + \mu \|\textbf{grad}\,(\phi_h)\|_{\mathbb{L}^2(\Omega)}^2 \lesssim \|\bm{u}_h\|_{A_\mu}^2,
\end{equation}
with notation
$$
\|\bm{u}_h\|_{A_\mu}^2= \|\textbf{curl} \,(\bm{u}_h)\|_{\mathbb{L}^2(\Omega)}^2 + \mu \|\bm{u}_h\|_{\mathbb{L}^2(\Omega)}^2.
$$
\end{proposition} 

\begin{proof}
Let $\bm{u}_h \in \bm{V}_h(\textbf{curl},\Omega)$. Using lemma \eqref{lem-semi-discrete-decomposition} we can find 
$\bm{\varphi} \in \mathbb{H}^1(\Omega)$ and $\phi_h \in V_h(\textbf{grad},\Omega)$ with the properties
\begin{equation}\label{pr-eq:semi-discrete-decomposition}
\begin{cases}
\bm{u}_h = \Pi_h^{\textbf{curl}} \bm{\varphi} + \textbf{grad}\,(\phi_h)\\
\|\bm{\varphi}\|_{\mathbb{H}^1(\Omega)} \lesssim \|\textbf{curl} \,(\bm{u}_h)\|_{\mathbb{L}^2(\Omega)}\\
\|\Pi_h^{\textbf{curl}} \bm{\varphi}\|_{\mathbb{L}^2(\Omega)} + \|\textbf{grad}\,(\phi_h)\|_{\mathbb{L}^2(\Omega)}\lesssim \|\bm{u}_h\|_{\mathbb{L}^2(\Omega)},
\end{cases}
\end{equation}
and let $\bm{\varphi}_h \in \mathbb{V}_h(\textbf{grad},\Omega)$ the stable approximation given by Lemma \ref{lem-stable-approximation-of-vect-grad}. Let us define
$$
\bm{w}_h=\Pi_h^{\textbf{curl}} (\bm{\varphi}-\bm{\varphi}_h).
$$
In this way, using the decomposition in \eqref{pr-eq:semi-discrete-decomposition}, we obtain
\begin{eqnarray*}
\bm{u}_h = \Pi_h^{\textbf{curl}} \bm{\varphi} + \textbf{grad}\,(\phi_h)&=&\Pi_h^{\textbf{curl}} (\bm{\varphi}-\bm{\varphi}_h)+ \Pi_h^{\textbf{curl}} \bm{\varphi}_h+ \textbf{grad}\,(\phi_h)\\
&=& \Pi_h^{\textbf{curl}} \bm{w}_h+ \Pi_h^{\textbf{curl}} \bm{\varphi}_h+ \textbf{grad}\,(\phi_h),
\end{eqnarray*}
and decomposition \eqref{eq:Hitpmair-Xu-decomposition} is proved. In order to show \eqref{eq:Hitpmair-Xu-decomposition-estimate}, we need to perform careful estimates. Indeed we have
\begin{eqnarray}\label{pr-eq:semi-discrete-decomposition-1}
h^{-1} \|\bm{w}_h\|_{\mathbb{L}^2(\Omega)} &=& h^{-1} \|\Pi_h^{\textbf{curl}} (\bm{\varphi}-\bm{\varphi}_h)\|_{\mathbb{L}^2(\Omega)}  \nonumber \\ 
& \lesssim & h^{-1}  \|\bm{\varphi}-\bm{\varphi}_h)\|_{\mathbb{L}^2(\Omega)}
 \lesssim  \|\bm{\varphi}\|_{\mathbb{H}^1(\Omega)} \lesssim \|\textbf{curl} \,(\bm{u}_h)\|_{\mathbb{L}^2(\Omega)},
\end{eqnarray}
where in the last estimate we have used the first inequality in \eqref{pr-eq:semi-discrete-decomposition}. Moreover, using the inverse inequality \eqref{eq:inverse-inequlity-curl} we get
\begin{equation}
\|\bm{w}_h\|_{\mathbb{L}^2(\Omega)} \lesssim h \|\textbf{curl} \,(\bm{u}_h)\|_{\mathbb{L}^2(\Omega)} \lesssim   \|\bm{u}_h\|_{\mathbb{L}^2(\Omega)}.
\end{equation}
Concerning the component $\bm{\varphi}_h$, we use first inequality in \eqref{pr-eq:semi-discrete-decomposition} to obtain
\begin{equation}
\|\bm{\varphi}_h\|_{\mathbb{H}^1(\Omega)} \lesssim \|\bm{\varphi}\|_{\mathbb{H}^1(\Omega)} \lesssim \|\textbf{curl} \,(\bm{u}_h)\|_{\mathbb{L}^2(\Omega)},
\end{equation}
and
\begin{equation}\label{pr-eq:semi-discrete-decomposition-4}
\|\bm{\varphi}_h\|_{\mathbb{L}^2(\Omega)} \leq \|\bm{\varphi}_h\|_{\mathbb{H}^1(\Omega)}  \lesssim  \|\bm{\varphi}\|_{\mathbb{H}^1(\Omega)}  \lesssim \|\textbf{curl} \,(\bm{u}_h)\|_{\mathbb{L}^2(\Omega)}.
\end{equation}
Combining \eqref{pr-eq:semi-discrete-decomposition-1}--\eqref{pr-eq:semi-discrete-decomposition-4} together with second estimate in \eqref{pr-eq:semi-discrete-decomposition}, and using the fact that $0 < \mu \leq 1$, we obtain the desired estimate  \eqref{eq:Hitpmair-Xu-decomposition-estimate}. This complete the proof of the proposition. 
\end{proof}

This last proposition leads to the following stable decomposition of $\bm{V}_h(\textbf{curl},\Omega)$, which is a discrete version of Theorem \ref{regular-decomposition-of-H(curl)}.

\begin{theorem}[Stable Hitpmair-Xu decomposition]\label{thm:stable-Hitpmair-Xu-decomposition}
For each $\bm{u}_h \in \bm{V}_h(\textbf{curl},\Omega)$, there exit $\bm{w}_h \in \bm{V}_h(\textbf{curl},\Omega)$, $\bm{\varphi}_h \in  \mathbb{V}_h(\textbf{grad},\Omega)$ and $\phi_h \in V_h(\textbf{grad},\Omega)$ such that   
\begin{equation}\label{eq:stable-Hitpmair-Xu-decomposition}
\bm{u}_h=\bm{w}_h+\Pi_h^{\textbf{curl}} \bm{\varphi}_h + \textbf{grad}\,(\phi_h),
\end{equation}
and 
\begin{equation}\label{eq:stable-Hitpmair-Xu-decomposition-estimate}
\|\bm{w}_h\|_{A_\mu}^2 + \|\bm{\varphi}_h\|_{\mathbb{H}^1(\Omega)}^2 + \mu \|\bm{\varphi}_h\|_{\mathbb{L}^2(\Omega)}^2 + \|\textbf{grad}\,(\phi_h)\|_{A_\mu}^2 \lesssim \|\bm{u}_h\|_{A_\mu}^2.
\end{equation}
\end{theorem}

\begin{proof}
The components $\bm{w}_h$, $\bm{\varphi}_h$ and $\phi_h$ are chosen as in Proposition \ref{prop:Hitpmair-Xu decomposition}. Hence, by remaking that 
$$
\|\textbf{grad}\,(\phi_h)\|_{A_\mu}^2= \|\textbf{curl} \,\left( \textbf{grad}\,(\phi_h) \right)\|_{\mathbb{L}^2(\Omega)}^2 + \mu \, \|\textbf{grad}\,(\phi_h)\|_{\mathbb{L}^2(\Omega)}^2 =\mu \|\textbf{grad}\,(\phi_h)\|_{\mathbb{L}^2(\Omega)}^2,
$$
we need only to estimate the first term in \eqref{eq:Hitpmair-Xu-decomposition-estimate}. For which, we have 
\begin{eqnarray*}
(h^{-2} + \mu) \, \|\bm{w}_h\|_{\mathbb{L}^2(\Omega)}^2 &=& h^{-2}\, \|\bm{w}_h\|_{\mathbb{L}^2(\Omega)}^2+ \mu \, \|\bm{w}_h\|_{\mathbb{L}^2(\Omega)}^2\\
& \gtrsim & \|\textbf{curl} \,(\bm{w}_h)\|_{\mathbb{L}^2(\Omega)}^2 + \mu \|\bm{w}_h\|_{\mathbb{L}^2(\Omega)}^2 = \|\bm{w}_h\|_{A_\mu}^2.
\end{eqnarray*}
This completes the proof.
\end{proof}

%{\color{red} \subsection{Discrete Decompositions with Mapping (To be discussed)}\label{subsec:asp-with-mapping}}

\subsection{Auxiliary Space Preconditioners}\label{subsec:construction-of-ASP}
Now we have all the ingredients to apply the abstract ASP theory of Section \ref{subsec:Auxiliary-Space-Preconditioning-Method}. Indeed, armed with the above stable discrete decomposition result, using the notations of Section \ref{subsec:Auxiliary-Space-Preconditioning-Method}, we consider $V=\bm{V}_h(\textbf{curl},\Omega)$ equipped with the bilinear form $a$ related to equation \eqref{eq:curl-variational-problem}, namely $a(\bm{w}_h,\widetilde{\bm{w}_h})= \left(\textbf{curl}\,(\bm{w}_h),\textbf{curl}\,(\widetilde{\bm{w}_h})\right)_{\mathbb{L}^2(\Omega)} + \mu (\bm{w}_h,\widetilde{\bm{w}_h})_{\mathbb{L}^2(\Omega)}$ for $\bm{w}_h, \widetilde{\bm{w}_h} \in V$, and auxiliary spaces $W_1= \mathbb{V}_h(\textbf{curl},\Omega)$ and $W_2=V_h(\textbf{grad},\Omega)$ equipped with the following inner products
\begin{equation*}
a_1(\bm{\varphi}_h,\widetilde{\bm{\varphi}_h})= (\bm{\varphi}_h,\widetilde{\bm{\varphi}_h})_{\mathbb{H}^1(\Omega)} + \mu (\bm{\varphi}_h,\widetilde{\bm{\varphi}_h})_{\mathbb{L}^2(\Omega)}, \quad \bm{\varphi}_h,\widetilde{\bm{\varphi}_h} \in W_1,
\end{equation*}
and
\begin{equation*}
a_2(\phi_h,\widetilde{\phi_h})= \mu \, (\textbf{grad}\,(\phi_h),\textbf{grad}\,(\widetilde{\phi_h}))_{\mathbb{L}^2(\Omega)},  \quad \phi_h,\widetilde{\phi_h} \in W_2,
\end{equation*}
respectively. The corresponding transfer operators are $\pi_1 = \Pi_h^{\textbf{curl}}\,:\, W_1 \longrightarrow V$ and $\pi_2=\textbf{grad}\,:\, W_2 \longrightarrow V$ and the smoother is simply the inner product $a$. 

Before we verify the validity of assumptions of Theorem \ref{th:ASP-lemma}, we transition to a matrix notation. So we write $C_h^1$ for the matrix related to the restriction of $\mathbb{H}^1(\Omega)$ inner product to $\mathbb{V}_h(\textbf{grad},\Omega)$, that is the matrix representation of the bilinear form 
\begin{equation*}
(\bm{\varphi}_h,\widetilde{\bm{\varphi}_h}) \in \mathbb{V}_h(\textbf{grad},\Omega) \times \mathbb{V}_h(\textbf{grad},\Omega)  \mapsto (\bm{\varphi}_h,\widetilde{\bm{\varphi}_h})_{\mathbb{H}^1(\Omega)}.
\end{equation*}
Similarly, we write $C_h^2$ for the matrix related to the restriction of the $\mathbb{L}^2(\Omega)$ inner product to $\mathbb{V}_h(\textbf{grad},\Omega)$. Let $L_h$ be the discrete Laplacian matrix, namely the matrix related to the mapping  
\begin{equation*}
(\bm{\phi}_h,\widetilde{\bm{\phi}_h}) \in V_h(\textbf{grad},\Omega) \times V_h(\textbf{grad},\Omega)  \longmapsto \left(\textbf{grad}\,(\bm{\phi}_h),\textbf{grad}\,(\widetilde{\bm{\phi}_h})\right)_{\mathbb{L}^2(\Omega)}.
\end{equation*}
We also write $P_h^{\textbf{curl}}$ and $G_h$  for the matrices related to the transform operators $\pi_1$ and $\pi_2$ respectively while $R_h$ further stands for the matrix related to the smoother. With these notations, a simple computation shows that ASP preconditioner  for problem \eqref{eq:curl-variational-problem} reads 
\begin{equation*}
B_h=R_h^{-1} + P_h^{\textbf{curl}}  \left(C_h^1+\mu C_h^2 \right)^{-1} \left( P_h^{\textbf{curl}}\right)^T + \mu^{-1} G_h L_h^{-1} \left(G_h\right)^T. 
\end{equation*}

A direct consequence of Theorem \ref{thm:stable-Hitpmair-Xu-decomposition} and Theorem \ref{th:ASP-lemma} the following fundamental result which show the mesh-independent of the preconditioner.
\begin{theorem}
For $0 < \mu \leq 1$, the spectral condition number $\kappa \left( B_h A_h\right)$ is bounded, with respect to $h$.
\end{theorem}






















\section{Algorithms}
% ...
\begin{minipage}{\textwidth}
  \begin{algorithm}[H]
  \DontPrintSemicolon
  \SetAlgoLined
  \SetKwInOut{Input}{Input}\SetKwInOut{Output}{Output}
  \Input{$\mathcal{A}^\tau, \mathcal{B}^\tau, b, u_0, \nu_1, \nu_{\ASP}$}
  \Output{$u_k$}
  \BlankLine

    $k \gets 0$\;
    \While{$k \leq \nu_{\ASP}$ \textbf{and} not convergence}{
      $u_k \gets \mbox{\texttt{smoother}}_1(\mathcal{A}^\tau,b,u_k,\nu_1)$ \tcp*{Apply Gauss-Seidel smoother}  
      $d_k \gets b - \mathcal{A}^\tau u_k$      \tcp*{Compute the defect}
      $u_c \gets \mathcal{B}^\tau d_k$          \tcp*{ASP correction}
      $u_k \gets u_k + u_c$      \tcp*{Update the solution}
      $k \gets k + 1$\;
    }

  \caption{{\sc ASP}: Auxiliary Space Preconditioning for $\bm{V}_h(\textbf{curl},\Omega)$}
  \label{algo:asp-1}
  \end{algorithm} 
\end{minipage}
% ...

\subsection{Fast Diagonalization method}
In order to device a fast solver the Poisson and Laplace problems, we choose to follow the work of Sangalli and Tani \cite{sangalli2016}, we describe in the sequel the fast diagonalization method in the case of Isogeometric Analysis. This method was first introduced in \cite{lynch1964}.
\\
For the sack of simplicity, we shall consider the following Laplace problem, 
\begin{equation}
  \begin{cases}
  - \nabla^2 u + \tau u = f, \quad \Omega \\ 
  u=0, \quad \partial\Omega
  \end{cases}
  \label{eq:laplace}
\end{equation}
The Poisson problem and its solver shall be retrieved with $\tau=0$.
After discretizing the Laplace problem, we get the following linear system
\begin{equation}
  \mathcal{L}_{\tau} x := \left( K_1 \otimes M_2 \otimes M_3 +  M_1 \otimes K_2 \otimes M_3 + M_1 \otimes M_2 \otimes K_3 + \tau M_1 \otimes M_2 \otimes M_3 \right) x = b 
  \label{eq:laplace-kron}
\end{equation}
We first consider the generalized eigendecompositions problems
\begin{equation}
  K_1 U_1 = M_1 U_1 D_1, \quad  
  K_2 U_2 = M_2 U_2 D_2, \quad 
  K_3 U_3 = M_3 U_3 D_3,
  \label{eq:generalized-eigen-decomp}
\end{equation}
where $D_1$, $D_2$ and $D_3$ are diagonal matrices such that
\begin{equation}
  U_1^T M_1 U_1 = I_1, \quad  
  U_2^T M_2 U_2 = I_2, \quad  
  U_3^T M_3 U_3 = I_3
%  \label{}
\end{equation}
Therefor, (\ref{eq:laplace-kron}) can be written as
\begin{equation}
  \left( U_1 \otimes U_2 \otimes U_3 \right)^{-1} 
  \left( D_1 \otimes I_2 \otimes I_3 +  I_1 \otimes D_2 \otimes I_3 + I_1 \otimes I_2 \otimes D_3 + \tau I_1 \otimes I_2 \otimes I_3 \right) 
  \left( U_1 \otimes U_2 \otimes U_3 \right)^{-T} 
  x = b 
  \label{eq:laplace-kron-fact}
\end{equation}
The direct solver for the Laplace problem (\ref{eq:laplace-kron}), is then given by the following algorithm, where we omit the initialization step achieved by solving the generalized eigendecompositions in (\ref{eq:generalized-eigen-decomp}):
% ...
\begin{algorithm}[ht]
\DontPrintSemicolon
\SetAlgoLined
\SetKwInOut{Input}{Input}\SetKwInOut{Output}{Output}
\Input{$\mathcal{L}_{\tau}, b$}
\Output{$x$}
\BlankLine
$\tilde{b} \gets \left( U_1 \otimes U_2 \otimes U_3 \right) b$ \; 
$\tilde{x} \gets \left( D_1 \otimes I_2 \otimes I_3 +  I_1 \otimes D_2 \otimes I_3 + I_1 \otimes I_2 \otimes D_3 + \tau I_1 \otimes I_2 \otimes I_3 \right)^{-1} \tilde{b}$ \; 
$x \gets \left( U_1 \otimes U_2 \otimes U_3 \right)^T \tilde{x}$ \; 

\caption{\texttt{fast\_diag}: Fast diagonalization method for Laplace problem}
\end{algorithm} 
% ...
We consider now the vector Laplace problem
\begin{equation}
  \begin{cases}
  - \nabla^2 \mathbf{u} + \tau \mathbf{u} = \mathbf{f}, \quad \Omega \\
  \mathbf{u}=0, \quad \partial\Omega
  \end{cases}
  \label{eq:laplace-vector}
\end{equation}
Which can be written in a matrix form as
\begin{equation} 
  \bm{\mathcal{L}}_{\tau} =
  \begin{bmatrix}
   \mathcal{L}_{\tau} &                  0 & 0  \\
                    0 & \mathcal{L}_{\tau} & 0  \\
                    0 &                  0 & \mathcal{L}_{\tau}  
  \end{bmatrix}.
  \label{eq:laplace-vector-matrix-form}
\end{equation}
Therefor, a fast solver for the vector Laplace problem (\ref{eq:laplace-vector}) can be given by the following algorithm
% ...
\begin{algorithm}[H]
\DontPrintSemicolon
\SetAlgoLined
\SetKwInOut{Input}{Input}\SetKwInOut{Output}{Output}
\Input{$\bm{\mathcal{L}}_{\tau}, b$}
\Output{$x$}
\BlankLine

  $b_1, b_2, b_3 \gets \texttt{unfold}(b)$ \; 
  $x_1 \gets \texttt{fast\_diag}(\mathcal{L}_{\tau}, b_1)$ \; 
  $x_2 \gets \texttt{fast\_diag}(\mathcal{L}_{\tau}, b_2)$ \; 
  $x_3 \gets \texttt{fast\_diag}(\mathcal{L}_{\tau}, b_3)$ \; 
  $x \gets \texttt{fold}(x_1, x_2, x_3)$ \; 

\caption{\texttt{fast\_diag}: Fast diagonalization method for the vector Laplace problem}
\end{algorithm} 
% ...

\subsection{Discrete derivatives}
We denote by $\matrixIdentity$ the identity matrix. 
The discrete derivatives in 2D are given by
\begin{align}
  \left\{ 
  \begin{array}{ll}
  \mathbb{G} &= 
    \begin{bmatrix}
      \mathcal{D}_1 \otimes \matrixIdentity_2
      \\
      \matrixIdentity_1 \otimes \mathcal{D}_2
    \end{bmatrix}
  \\
  \\
  \pmb{\mathbb{C}} &= 
    \begin{bmatrix}
      \matrixIdentity_1 \otimes \mathcal{D}_2
      \\
    - \mathcal{D}_1 \otimes \matrixIdentity_2
    \end{bmatrix} 
  \\
  \\
  \mathbb{C} &= 
    \begin{bmatrix}
    - \matrixIdentity_1 \otimes \mathcal{D}_2
     & 
      \mathcal{D}_1 \otimes \matrixIdentity_2
    \end{bmatrix} 
  \\
  \\
  \mathbb{D} &= 
    \begin{bmatrix}
      \mathcal{D}_1 \otimes \matrixIdentity_2
     & 
      \matrixIdentity_1 \otimes \mathcal{D}_2
    \end{bmatrix}
  \end{array} \right.
  \label{eq:discrete-derivatives-2d}
\end{align}
The discrete derivatives in 3D are given by
\begin{align}
  \left\{ 
  \begin{array}{ll}
  \mathbb{G} &= 
    \begin{bmatrix}
      \mathcal{D}_1 \otimes \matrixIdentity_2 \otimes \matrixIdentity_3
      \\
      \matrixIdentity_1 \otimes \mathcal{D}_2 \otimes \matrixIdentity_3 
      \\
      \matrixIdentity_1 \otimes \matrixIdentity_2 \otimes \mathcal{D}_3
    \end{bmatrix}
  \\
  \\
  \mathbb{C} &= 
  \begin{bmatrix}
    0    &    - \matrixIdentity_1 \otimes \matrixIdentity_2 \otimes \mathcal{D}_3  &  \matrixIdentity_1 \otimes \mathcal{D}_2 \otimes \matrixIdentity_3 
    \\
    \matrixIdentity_1 \otimes \matrixIdentity_2 \otimes \mathcal{D}_3   &    0   &   - \mathcal{D}_1 \otimes \matrixIdentity_2 \otimes \matrixIdentity_3 
    \\
    - \matrixIdentity_1 \otimes \mathcal{D}_2 \otimes \matrixIdentity_3  & \mathcal{D}_1 \otimes \matrixIdentity_2 \otimes \matrixIdentity_3 & 0 
  \end{bmatrix} 
  \\
  \\
  \mathbb{D} &= 
    \begin{bmatrix}
      \mathcal{D}_1 \otimes \matrixIdentity_2 \otimes \matrixIdentity_3
      & 
      \matrixIdentity_1 \otimes \mathcal{D}_2 \otimes \matrixIdentity_3 
      &
      \matrixIdentity_1 \otimes \matrixIdentity_2 \otimes \mathcal{D}_3
    \end{bmatrix}
  \end{array} \right.
  \label{eq:discrete-derivatives-3d}
\end{align}
\todo{
\begin{remark}
The actual implementation is based on a matrix form, but we should avoid it and implement these operators as functions. This will reduce the memory usage.  
\end{remark}
}
\subsection{The Histopolation Operator}
In the 1D case, the DeRham sequence involves two spaces $V_0 := \Vgrad$ and $V_1 := \Ltwo$. We are interested in the restriction of the histopolation operator to functions in $V_0$. We consider a function $u_h \in V_0$ that is expanded as $u_h := \sum\limits_{1 \le j \le n_{V_0}} u_j \Njone$. 

Let us first define the histopolation matrix form $\mathcal{H}$ as

\subsection{The $\Picurl$ Operator}
\todo{TODO}

In the 2D case, the $\Picurl$ is defined as 
\begin{align}
  P_h^{\textbf{curl}} &=
  \begin{bmatrix}
      \mathcal{H}_1 \otimes \matrixIdentity_2
      \\
      \matrixIdentity_1 \otimes \mathcal{H}_2
  \end{bmatrix}
%  \label{}
\end{align}

In the 3D case, the $\Picurl$ is defined as 
\begin{align}
    \Picurl  & = 
    \begin{bmatrix}
      \mathcal{H}_1 \otimes \matrixIdentity_2 \otimes \matrixIdentity_3
      \\
      \matrixIdentity_1 \otimes \mathcal{H}_2 \otimes \matrixIdentity_3 
      \\
      \matrixIdentity_1 \otimes \matrixIdentity_2 \otimes \mathcal{H}_3 
    \end{bmatrix}
  \label{eq:commuting-projectors-splines-3d}
\end{align}


\subsection{Triangular solver and the symmetric Gauss-Seidel method}
We consider in the following a block matrix 
%
\begin{equation} %\label{eq:matr_A_2d}
\mathcal{A} =\begin{bmatrix}
 A_{11} & A_{12} & A_{13}  \\
 A_{21} & A_{22} & A_{23}  \\
 A_{31} & A_{22} & A_{33}  
\end{bmatrix}.
\end{equation}
%

% ...
\begin{algorithm}[H]
\DontPrintSemicolon
\SetAlgoLined
\SetKwInOut{Input}{Input}\SetKwInOut{Output}{Output}
\Input{$\mathcal{A}, x, b, \nu_1$}
\Output{$x$}
\BlankLine

  \For{$i \gets 1$ \textbf{to} $\nu_1$} {
    $x \gets x + \texttt{spsolve}(\mathcal{A}, b - \mathcal{A} x, \texttt{lower}=\texttt{True})$ \; 
  }
  \For{$i \gets 1$ \textbf{to} $\nu_1$} {
    $x \gets x + \texttt{spsolve}(\mathcal{A}, b - \mathcal{A} x, \texttt{lower}=\texttt{False})$ \; 
  }

\caption{\texttt{gauss\_seidel}: Symmetric Gauss Seidel solver}
\end{algorithm} 
% ...

% ...
\begin{algorithm}[H]
\DontPrintSemicolon
\SetAlgoLined
\SetKwInOut{Input}{Input}\SetKwInOut{Output}{Output}
\Input{$\mathcal{A}, b$}
\Output{$x$}
\BlankLine

  $b_1, b_2, b_3 \gets \texttt{unfold}(b)$ \; 
  $x_1 \gets \texttt{spsolve}(A_{11}, b_1, \texttt{lower}=\texttt{True})$                                    \tcp*{Comment}
  $x_2 \gets \texttt{spsolve}(A_{22}, b_2 - A_{21} x_1, \texttt{lower}=\texttt{True})$                       \tcp*{Comment}
  $x_3 \gets \texttt{spsolve}(A_{33}, b_3 - A_{31} x_1 - A_{32} x_2, \texttt{lower}=\texttt{True})$          \tcp*{Comment}
  $x \gets \texttt{fold}(x_1, x_2, x_3)$ \; 

\caption{\texttt{spsolve}: Triangular solver for lower block matrix}
\end{algorithm} 
% ...

% ...
\begin{algorithm}[H]
\DontPrintSemicolon
\SetAlgoLined
\SetKwInOut{Input}{Input}\SetKwInOut{Output}{Output}
\Input{$\mathcal{A}, b$}
\Output{$x$}
\BlankLine

  $b_1, b_2, b_3 \gets \texttt{unfold}(b)$ \; 
  $x_3 \gets \texttt{spsolve}(A_{33}, b_3, \texttt{lower}=\texttt{False})$                                    \tcp*{Comment}
  $x_2 \gets \texttt{spsolve}(A_{22}, b_2 - A_{23} x_3, \texttt{lower}=\texttt{False})$                       \tcp*{Comment}
  $x_1 \gets \texttt{spsolve}(A_{11}, b_1 - A_{12} x_2 - A_{13} x_3, \texttt{lower}=\texttt{False})$          \tcp*{Comment}
  $x \gets \texttt{fold}(x_1, x_2, x_3)$ \; 

\caption{\texttt{spsolve}: Triangular solver for upper block matrix}
\end{algorithm} 
% ...

Since the diagonal block matrices can be either a Kronecker product of 3 matrices or the sum a Kronecker product of 3 matrices, we can then derive efficient matrix-free implementation as described in the following algorithms.

% ...
\begin{algorithm}[ht]
\DontPrintSemicolon
\SetAlgoLined
\SetKwInOut{Input}{Input}\SetKwInOut{Output}{Output}
\Input{$\mathcal{A} = A_1 \otimes A_2 \otimes A_3, b$}
\Output{$x$}
\BlankLine

  \For{$i_1 \gets 1$ \textbf{to} $n_1$} {
    \For{$i_2 \gets 1$ \textbf{to} $n_2$} {
      \For{$i_3 \gets 1$ \textbf{to} $n_3$} {
        $i \gets \texttt{multi\_index}(i_1, i_2, i_3)$ \; 
        $r \gets 0$ \; 
        $a_d \gets 1$ \; 

        \For{$k_1 \gets A_1.\texttt{ptr}[i_1]$ \textbf{to} $A_1.\texttt{ptr}[i_1+1] - 1$} {
          $j_1 \gets A_1.\texttt{indices}[k_1]$ \; 
          $a_1 \gets A_1.\texttt{data}[k_1]$ \; 
          \For{$k_2 \gets A_2.\texttt{ptr}[i_2]$ \textbf{to} $A_2.\texttt{ptr}[i_2+1] - 1$} {
            $j_2 \gets A_2.\texttt{indices}[k_2]$ \; 
            $a_2 \gets A_2.\texttt{data}[k_2]$ \; 
            \For{$k_3 \gets A_3.\texttt{ptr}[i_3]$ \textbf{to} $A_3.\texttt{ptr}[i_3+1] - 1$} {
              $j_3 \gets A_3.\texttt{indices}[k_3]$ \; 
              $a_3 \gets A_3.\texttt{data}[k_3]$ \; 
              $j \gets \texttt{multi\_index}(j_1, j_2, j_3)$ \; 
              \uIf{$i < j$}{
                $r \gets r + a_1 a_2 a_3 x[j]$ \; 
              }
              \Else{
                $a_d \gets a_1 a_2 a_3$ \; 
              }
            }
          }
        }
        $x[i] \gets \frac{1}{a_d}(b[i] - r)$ \; 
      }
    }
  }

\caption{\texttt{spsolve}: Triangular solver for lower Kronecker product matrix, for CSR storage.}
\end{algorithm} 
% ...


% ...
\begin{algorithm}[ht]
\DontPrintSemicolon
\SetAlgoLined
\SetKwInOut{Input}{Input}\SetKwInOut{Output}{Output}
\Input{$\mathcal{A} = A_1 \otimes A_2 \otimes A_3, b$}
\Output{$x$}
\BlankLine

  \For{$i_1 \gets 1$ \textbf{to} $n_1$} {
    \For{$i_2 \gets 1$ \textbf{to} $n_2$} {
      \For{$i_3 \gets 1$ \textbf{to} $n_3$} {
        $i \gets \texttt{multi\_index}(i_1, i_2, i_3)$ \; 
        $r \gets 0$ \; 
        $a_d \gets 1$ \; 

        \For{$k_1 \gets A_1.\texttt{ptr}[i_1]$ \textbf{to} $A_1.\texttt{ptr}[i_1+1] - 1$} {
          $j_1 \gets A_1.\texttt{indices}[k_1]$ \; 
          $a_1 \gets A_1.\texttt{data}[k_1]$ \; 
          \For{$k_2 \gets A_2.\texttt{ptr}[i_2]$ \textbf{to} $A_2.\texttt{ptr}[i_2+1] - 1$} {
            $j_2 \gets A_2.\texttt{indices}[k_2]$ \; 
            $a_2 \gets A_2.\texttt{data}[k_2]$ \; 
            \For{$k_3 \gets A_3.\texttt{ptr}[i_3]$ \textbf{to} $A_3.\texttt{ptr}[i_3+1] - 1$} {
              $j_3 \gets A_3.\texttt{indices}[k_3]$ \; 
              $a_3 \gets A_3.\texttt{data}[k_3]$ \; 
              $j \gets \texttt{multi\_index}(j_1, j_2, j_3)$ \; 
              \If{$i \ge j$}{
                $r \gets r + a_1 a_2 a_3 x[j]$ \; 
              }
              \If{$i = j$}{
                $a_d \gets a_1 a_2 a_3$ \; 
              }
            }
          }
        }
        $x[i] \gets \frac{1}{a_d}(b[i] - r)$ \; 
      }
    }
  }

\caption{\texttt{spsolve}: Triangular solver for upper Kronecker product matrix, for CSR storage.}
\end{algorithm} 
% ...

\subsection{Computational Cost}

\newpage
\section{Numerical Results}


% ********************************************

% ********************************************
\clearpage
\newpage
\bibliographystyle{plain}
\bibliography{references}
% ********************************************

% ********************************************
%\clearpage
%\newpage
%\section*{Appendix I}


% ********************************************

\end{document}


